\documentclass[a4paper]{jsarticle}
\usepackage{wabun}

\begin{document}
\yourclass{\raisebox{14pt}{情報システム基盤学}}
\yournum{\raisebox{14pt}{1353015}}
\yourname{鈴木 駿介}
\yourtitle{ロールベースのP2Pライブストリーミングアーキテクチャの研究}
\youryoushi{
\large
\begin{center}
\begin{minipage}{15cm}
\hspace*{1zw}ネットワーク技術の発展に伴い, 近年ではリアルタイム動画配信サービスが人気となっている. このようなサービスはサーバから各クライアントに配信する, サーバ-クライアント方式が一般的であるが, 配信サーバのコスト削減などの理由で, サーバを介さずに直接クライアント同士で配信を行うP2Pを利用したライブストリーミング配信が期待されている. しかし, 既存のP2Pライブストリーミングシステムでは, 途中参加したユーザはそれまでの配信内容を把握することが出来ないといった問題がある. そこで本研究ではP2Pライブストリーミングにおいて, 途中参加したユーザがダイジェスト視聴可能なP2Pライブストリーミングシステムを提案する. 本研究では2つの段階を考えた.
\\
\hspace*{1zw}1つ目の段階ではダイジェスト生成方式として, 3つの方式を提案した. 提案するシステムは参加ユーザが自由にコメント投稿出来ることを想定し, 3つの方式ともコメントをした参加ユーザ数を利用した. 実験では, ある動画に対して予め正解シーンを用意し, 提案方式がダイジェストとして選出したシーンとその正解シーンとの適合率で評価を行った. その結果「最小二乗法に基づくダイジェスト生成方式」が最も適合率が高く, 映像全体の66\%以上のダイジェストを判定出来ることが分かり, 適切な方式であることがわかった.
\\
\hspace*{1zw}2つ目の段階ではトポロジを提案した. 具体的には複数クラスタ型において, 各ノードに役割をもたせたトポロジを設計した. 役割として「ゲートノード」, 「セミゲートノード」, 「ダイジェストノード」を用意した. ダイジェスト生成方式ではコメントをしたユーザ数を利用したが, トポロジ設計では各ノードの役割を決定するのに各コメント数におけるユーザ数の割合を利用した. ゲートノードは配信者ノードから配信されたコンテンツのパケットをクラスタ内で一番最初に取得する役目を担う. 高帯域かつコメント数の多いノードを選出した. セミゲートノードはクラスタ内でゲートノードから受信したパケットをクラスタ内部に拡散する役目を担う. ゲートノードの次に高帯域でコメント数の多いノードを選出した. ダイジェストノードはダイジェストの作成, また作成したダイジェストを保有し, 新規参加ノードへ送信する役目を担う. 最もコメント数の多いノードを選出した. 実験ではNS-2というシミュレーションを使って評価を行った. 役割の適切性に対する評価では, 役割を与えない場合ではノード数が増えるに従ってスループットの値が下がっていくが, 役割を与えた場合ではスループットの値が下がること無く, ノード数が増えても継続的に高いスループットを示しており, 役割を与えることの有用性を確認することが出来た.
\end{minipage}
\end{center}
}
\writeall

\end{document}
