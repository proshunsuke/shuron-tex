\documentclass[a4paper,12pt]{jsbook}
\usepackage{fancyhdr} %デザイン設定
\usepackage{shuron} %修論用スタイル
\usepackage{url} %URL表示用
\usepackage[dvipdfmx]{graphicx} %画像ファイル用
\usepackage{flafter} %図の位置が図参照より後になるように
\usepackage{subfigure} %部分図
\usepackage{mediabb} %PDFファイルを画像として参照
\usepackage{listings,jlisting} %プログラムコード用
\usepackage{amsmath, amssymb} %オーダ記号などの数学記号
%\usepackage{theorem} %定義
\usepackage{multirow} %縦に跨がる表
\usepackage{multicol} %段落
\usepackage{algorithm}
%% \usepackage{algorithmic}
\usepackage{algcompatible}
\usepackage[dvipdfmx]{hyperref}
\usepackage{pxjahyper}

\renewcommand{\lstlistingname}{プログラム}
\renewcommand{\lstlistlistingname}{プログラムコード一覧}
\lstset{language=Java,
        basicstyle=\footnotesize,
        commentstyle=\textit,
        classoffset=1,
        keywordstyle=\bfseries,
         showstringspaces=false,
        tabsize=4,
        frame=lines,
        numbers=left,
        numberstyle=\scriptsize,
        numbersep=5pt,
        escapechar=\#
}
%\theoremstyle{break}
%\newtheorem{define}{定義}
%\newcommand{\transprog}{$ P \xrightarrow{\mathcal{T}} P'$}

\newcommand{\textref}[1]{\ref{#1}節(p.\pageref{#1})}
\newcommand{\figref}[1]{\figurename~\ref{#1}}
\newcommand{\tblref}[1]{\tablename~\ref{#1}}
\newcommand{\cdref}[1]{\lstlistingname~\ref{#1}}
\newcommand{\chapterref}[1]{\prechaptername \ref{#1} \postchaptername}

\def\PT{$\mathsf{P^T}$}
\def\PF{$\mathsf{P^F}$}

\makeatletter
  \def\@cite#1{\textsuperscript{[#1]}}
  \renewcommand{\ALG@name}{変換規則}
\makeatother

\voffset=-10.5mm

\begin{document}
\setcounter{page}{0}
\thispagestyle{empty}
\noindent
\begin{tabular}{c}
{\ueclogo B} \vspace{1.5cm}     \\
{\Large 平成26年度 修士論文}    \\
\end{tabular}

\vspace{2.0cm}

\begin{center}
{\huge \bf ロールベースのP2Pライブ \\ ストリーミングアーキテクチャの研究}
\end{center}

\vspace{3cm}

\LARGE
\begin{flushright}
電気通信大学~大学院情報システム学研究科         \\
情報システム基盤学専攻                          \\
1353015~~鈴木 駿介                              \\

\vspace{2cm}

{\def\arraystretch{0.6}
\begin{tabular}{rll@{}}
指導教員        & 多田 好克     & 教授                \\
                & 末田 欣子     & 客員准教授                  \\
                & 古賀 久志     & 准教授                \\
                                                        \\
提出日          & \multicolumn{2}{c@{}}{平成27年1月26日}        \\
\end{tabular}
}
\end{flushright}

\normalsize
\newpage

\setcounter{tocdepth}{2}
\pagenumbering{roman}
\tableofcontents
%\setcounter{page}{1}

\listoffigures

\listoftables

%% \lstlistoflistings

%%%%%%%%%%%%%%%%%%%%%%%%%%%%%%%%%%%%%%%%%%%%%%%%%%%%%%%%%%%%%%%%%%%%%%
\chapter{はじめに}
\pagenumbering{arabic}
ここにアブストを書く。

背景、目的、提案手法、実験と結果、今後


\clearpage
\chapter{背景} \label{txt:introductions}
ネットワーク技術の発達に伴い, ネットワークを用いた動画配信サービスが人気である. 特にリアルタイム動画配信サービスが人気であり, 日本ではニコニコ生放送\cite{nico}やTwitCasting\cite{twi}, アメリカではUstream\cite{ust}, 韓国ではAfeecaTV\cite{afr}と, 近年世界中で急速に普及している.  このようなサービスの形態としては, サーバから配信された映像ををクライアントが視聴する, サーバ-クライアント方式が一般的である. しかし, 配信サーバのコスト削減や配信者の負荷を軽減するため, サーバを介さず直接クライアント同士で配信を行うP2P(peer to peer)を利用したライブストリーミング配信が期待されている.

P2Pライブストリーミングは主にアプリケーションレベルマルチキャスト(ALM: Application Level Multicast)によって行われる. ALMはアプリケーションによって実現されるため開発が容易で多くの研究がなされている.

%% そこで本研究では, 配信に途中参加したユーザがダイジェスト視聴可能なP2Pライブストリーミングシステムにおいて, P2Pネットワーク内でダイジェストを生成し, 保持し, 広めるために各ピアが役割を持ったトポロジを提案する. ダイジェストを作成する専用のサーバは設置せず, ネットワーク内に存在するピアが作成し, それをネットワーク内に広める. システムはダイジェストがP2Pネットワーク内で枯渇しないような設計にする必要がある.

\section{P2Pライブストリーミングのトポロジ}
ALMは主にツリー型とメッシュ型に分類される. ツリー型は遅延が少なく構築が容易であるというメリットがあるが, 耐故障性やノードの離脱に弱いといったデメリットがある. メッシュ型は耐故障性やノードの離脱に強いというメリットがあるが, 複数の経路を用いるため遅延が大きくなるといったデメリットがある. 一方でこれらツリー型やメッシュ型の欠点を補うために複数クラスタ型\cite{dis},\cite{streamline}が提案されている. 複数クラスタ型は複数のクラスタを構築し各サブストリームをそれぞれのクラスタで配信するといった方法である. 複数クラスタ型ではツリー構造のように中継ノードにストリームを渡すため, 配信の負荷が軽減される. またクラスタ内部では各ノードが複数の経路を持つため耐故障性に優れる. 以下に複数クラスタ型の既存研究をあげる.

\newpage

\subsection{階層型クラスタ構造}
Yang Guoらは階層的クラスタHCPS(Hierarchically Clus-tered P2P Video Streaming)\cite{hcps}\ref{fig:hcps}を提案している. HCPSはクラスタ間で帯域のバランスを取ることにより, より品質の高い映像を流すことを可能にしている. また, 隣のピアへのチャンクの受け渡しの際に, アップリンク帯域幅を有効活用出来るようなスケジューリングアルゴリズムを提案している. クラスタ内部は完全結合となっており, 各クラスタのアップロード容量が均一になるように構成されている. クラスタを形成する際にクラスタ内で一番最初に配信内容を受け取るノードをヘッドノードと定義し, そのアップロード容量は大きいものを選出している.

HCPSはクラスタの中身が完全結合のため各ノードへのホップ数や遅延が多くなってしまうという課題がある.

\begin{figure}
  \centering
  \includegraphics[width=1\hsize]{fig/hcps.eps}
  \caption{階層型クラスタHCPSのトポロジ設計}
  \label{fig:hcps}
\end{figure}

\subsection{重畳クラスタ木型動画配信システム}
元橋らは重畳クラスタ木方式の動画配信システム\cite{chojo}\ref{fig:chojo}を提案している. 階層的なクラスタ構造になっており, クラスタ間の配信木の中継ノードであるゲートノードが存在する. さらにゲートノードはクラスタ内で一番最初に配信内容を受け取り, それをクラスタ内に広める役割を担っている. ゲートノード選出方式として, 滞在時間の長さとRTT(Round Trip Time)を考慮した方法を提案している. 離脱耐性向上型, 配信遅延抑制型, ハイブリッド型の実験をしたところ, ハイブリッド型が最も性能が良いという結果が出ている.

重畳クラスタ木方式は全体として木構造のため下位のクラスタほど配信者からのホップ数が大きくなってしまうという課題がある.

\begin{figure}
  \centering
  \includegraphics[width=1\hsize]{fig/chojo.eps}
  \caption{重畳型クラスタ木型動画配信システムのトポロジ設計}
  \label{fig:chojo}
\end{figure}

\newpage

\subsection{ハイブリッドアーキテクチャ構造}
Huey-Ing Liuらは局所性と貢献度を考慮したハイブリッドアーキテクチャであるMeTree\cite{metree}を提案している. MeTreeはISP(Internet Service Provider)ごとにクラスタリングを行いそれぞれのノードをメッシュで接続し, 生成された各クラスタ同士をツリー構造で接続している. ツリー構造とメッシュ構造のハイブリッド構造となっている. 配信内容をより多くのノードに広めるような貢献度の高いピアには質の高い映像を配信し, 逆に貢献度の低いピアには質の低い映像を配信する. 異なる貢献度を持つピアに異なるQoE(Quality of Experience)を与えている. 物理トポロジとオーバーレイを構築するためのピアの貢献度の両方を考慮することにより, 遅延を減少させている.

MeTreeでは貢献度を意識した設計のため貢献度の低いノードは良い映像が見られず, ネットワーク全体のQoEは低下してしまうという課題がある.

\begin{figure}
  \centering
  \includegraphics[width=1\hsize]{fig/metree.eps}
  \caption{Metreeのトポロジ設計}
  \label{fig:metree}
\end{figure}

\newpage

\section{ダイジェスト生成方式}
一方で, 動画配信サービスにおいてダイジェストを見ることは, 動画全体の雰囲気を知るために有用である. リアルタイム動画配信であるライブストリーミングにおいても, ユーザが途中から配信に参加した場合にそれまでの配信の内容を把握出来るという点においてダイジェストを見ることは有用である. しかし, P2Pのライブストリーミングにおいては, サーバ-クライアント方式と比べてダイジェストを保存しておくサーバを用意することが出来ないといった問題があり, 既存のサーバ-クライアント方式のためのダイジェスト生成方式が適応出来ない. 以下にダイジェスト生成方式の既存研究をあげる.

\subsection{動画に対するダイジェスト生成方式}
橋本らはスポーツ映像を対象として, 映像メタデータと利用者の嗜好情報を利用したパーソナルダイジェスト生成方式PDMS(Personal Digest Making Scheme)\cite{pdms}を提案している. PDMSは, 映像メタデータから発生事象の重要度を自動的に検出し, 複数のダイジェストを選択する. 従来の手動によるコンテンツ作成に比べ, 携帯端末向けに低コストで生成することができる. 野球の映像を対象とし, 事前に選んだ正解集合と比較して適合率によってダイジェスト配信システムの重要度算出アルゴリズムの評価を行った.

PDMSでは一度映像を整理してシーン毎に分類し, その上でダイジェストシーンを選出している. そのためライブ映像に対してはPDMSの手法を適応することが出来ない.

\subsection{リアルタイム動画に対するダイジェスト生成方式}
熊野らは野球の実況中継映像を対象として, 自動的にインデックス情報を付与して, ハイライトシーンを検出する, リアルタイムダイジェスト生成システム\cite{yakyu}を提案している. システムでは特に野球のPC(Picher and Catcher)シーンを画像解析により検出し, さらに音声解析により特別なイベントと判断されたキーワードを含む区間をハイライトシーンとして生成している. 実験では予め正解であるシーンを用意し, システムを適応した際の適合率により評価を行った. 結果は最も高い適合率で97.2\%という結果であり, 有用であることを示していた.

熊野らの研究ではライブ映像に対してダイジェストをリアルタイムに生成している点が優れている. しかし, 野球という特定の分野の映像を対象にしていたため, 様々な内容の動画配信に対応させることは難しい.


%% \clearpage
%% \chapter{関連研究} \label{txt:related}
%% \input{tex/related.tex}

\clearpage
\chapter{提案手法} \label{txt:proposed}
ここに提案手法を書く、最初に研究の目的を示す

まず最初にダイジェスト生成方式、次にトポロジ設計
\section{ダイジェスト生成方式}
ダイジェスト生成方式について述べる

3つ提案した

いづれも突出点を発見する手法

\subsection{ダイジェスト生成方式に対する要求条件}
要求条件を提示する

\subsection{閾値に基づくダイジェスト生成方式}
具体的な方法

式

グラフ

\subsection{前後比較に基づくダイジェスト生成方式}
具体的な方法

式

グラフ


\subsection{最小二乗法に基づくダイジェスト生成方式}
具体的な方法

式

グラフ

\subsection{既存のダイジェスト生成方式との比較}

既存手法との定性評価

\section{トポロジ設計}
ダイジェストを生成し、保持し、広めるためのトポロジ設計

\subsection{トポロジ設計に対する要求条件}
要求条件を提示する

\subsection{配信内容に対するコメント}
コメントの役割を書く

\subsection{ノードの役割}
\subsubsection{ゲートノード}
ゲートノードについて

\subsubsection{セミゲートノード}
セミゲートノードについて

\subsubsection{ダイジェストノード}
ダイジェストノードについて

\subsubsection{その他のノード}
トラッカーサーバの存在とノーマルノードについて

\subsection{ノードの役割決定方法}
各役割についての決定方法を書く

役割決定の図

\subsection{ノード間接続方法}

図を説明しながらノード間の接続方法について書く

クラスタ間の図

\subsection{新規参加ピアについて}
新規参加ピアの行動について書く

新規参加ピアの行動のシーケンス図

\subsection{再構築のタイミング}
再構築のタイミング3パータンを書く

場合によっては図を用いる

\subsection{既存のトポロジ設計との比較}

既存手法との定性評価




\clearpage
\chapter{評価} \label{txt:eval}
本研究では,  配信に途中参加したユーザがダイジェスト視聴可能なP2Pライブストリーミングシステムを提案してきた. そのために2つの段階を考え, 1つ目の段階では, P2Pネットワーク内で特別なサーバに頼らない, 汎用的なダイジェスト生成方式を提案した. 2つ目の段階では, 1つ目の段階において生成されたダイジェストをP2Pネットワーク内で保持し, それを拡散させるためのトポロジ設計を提案した.

この章では, まず1段階目のダイジェスト生成方式に対する評価を行い, その次に2段階目のトポロジ設計に対する評価を行う.

ダイジェスト生成方式では, まず3つの提案方式それぞれの適切な閾値を決めるための評価を行う. その次に実際の動画コンテンツを対象に提案システムを適応し, ダイジェスト生成に最も適切な方式を決定する. トポロジ設計では, まず各役割の適切な割合を決めるための評価を行う. その次にその役割が適切な役目を果たしているかの評価を行う. さらに生成されたダイジェストが適切にP2Pネットワーク内で保持され, 視聴出来たかの評価を行う.

表\ref{tbl:env}に評価実験を行った環境を示す.

\begin{table}[h]
  \caption{実行環境}
  \label{tbl:env}
  \centering
      {\small
        \begin{tabular}{|l|l|l|} \hline
          環境名 & 規格 & バージョン \\ \hline \hline
          OS & Ubuntu & 12.04 64-bit  \\ \hline
          CPU & Intel Core i7 2.10~GHz & \\ \hline
          メモリ & 8~GB & \\ \hline
          シミュレータ & NS2 & 2.35 \\ \hline
          言語 & OTcl & 1.14 \\ \cline{2-3}
           & TK & 8.5.10 \\ \hline
          可視化ツール & nam & 1.15 \\ \hline
        \end{tabular}
      }
\end{table}

\section{ダイジェスト生成方式に対する評価}
\subsection{評価の準備}
ダイジェスト生成方式に対する評価を行う際の準備について述べる. 評価の1つ目は, 3つの提案方式の適切な閾値を決めることである. 2つ目は3つの提案方式のうち最も適切な方式を決定することである.

2つの評価に共通するものとして, 対象となる映像がある. まず時間の長さの異なる3つ映像に対して, 時間の10分の1となる1つ2分間の正解シーンを用意した. つまり, 例えば100分の映像を扱う際には5個の正解シーンが存在するということである.

\subsection{適切な閾値の決定}
3つの提案方式の適切な閾値を決定するための評価方法について述べる. まず, 提案システムがダイジェストであると選出したシーンの数が, 元々用意しておいた正解シーンよりも半分以上であるものを集計する. なお, 正解シーンの前後2分間をダイジェストと判断した場合を選出出来たと判断して集計する. 集計した後, 最も多く正解シーンを選出した閾値を適切な値とする. 例えば100分の映像ならば, 正解シーンが5つ存在する. そのうち提案システムが半分以上選出出来たらならばプラス1カウントする. これを3つの映像それぞれについて行い, 最も多く選出出来た(映像は3種類なので最大で3)閾値を適切な値とする.

\subsubsection{閾値に基づくダイジェスト生成方式の閾値}
「閾値に基づくダイジェスト生成方式」では式\ref{fig:sikiiti}において閾値は, 増加率($Th_{inc}$), 減少率($Th_{dec}$), 一定期間($T$)であった. 図\ref{fig:digest1-1}は閾値に基づくダイジェスト生成方式の抽出例である. 時刻に付いている赤い丸が正解シーンであり, 縦に緑の線がシステムがダイジェストであると判断して選出した箇所である. 図を見ると2箇所正解していることがわかる.

表\ref{tbl:digest1-1}に増加率0.05の時の正解シーン選出の結果を, 表\ref{tbl:digest1-2}に増加率0.10の時の正解シーン選出の結果を示す. 実験の結果, 増加率が0.1, 減少率が0.03, 一定期間が3である時が一番正解数が多く, 適切な値であることがわかった.

\begin{figure}[h]
  \centering
  \includegraphics[width=1\hsize]{fig/digest1-1.eps}
  \caption{閾値に基づくダイジェスト生成方式の抽出例}
  \label{fig:digest1-1}
\end{figure}

\begin{table}[h]
  \caption{増加率0.05の時の正解シーン選出の結果}
  \label{tbl:digest1-1}
  \centering
      {\small
        \begin{tabular}{|l|l||l|l|l|} \hline
           \multicolumn{5}{|c|}{増加率($Th_{inc}$)0.05} \\ \hline
           & & \multicolumn{3}{c|}{減少率($Th_{dec}$)} \\ \hline
           & & 0.03 & 0.05 & 0.10 \\ \hline \hline \cline{2-5}
           & 1 & 0 & 1 & 1 \\ \cline{2-5}
           & 2 & 0 & 0 & 2 \\ \cline{2-5}
           & 3 & 2 & 2 & 1 \\ \cline{2-5}
           一定期間($T$)& 4 & 1 & 1 & 1 \\ \cline{2-5}
           & 5 & 2 & 2 & 1 \\ \cline{2-5}
           & 6 & 2 & 2 & 1 \\ \cline{2-5}
           & 7 & 1 & 1 & 1 \\ \cline{2-5}
           \hline
        \end{tabular}
      }
\end{table}

\newpage

\begin{table}[h]
  \caption{増加率0.10の時の正解シーン選出の結果}
  \label{tbl:digest1-2}
  \centering
      {\small
        \begin{tabular}{|l|l||l|l|l|} \hline
           \multicolumn{5}{|c|}{増加率($Th_{inc}$)0.10} \\ \hline
           & & \multicolumn{3}{c|}{減少率($Th_{dec}$)} \\ \hline
           & & 0.03 & 0.05 & 0.10 \\ \hline \hline \cline{2-5}
           & 1 & 1 & 1 & 0 \\ \cline{2-5}
           & 2 & 2 & 2 & 1 \\ \cline{2-5}
           & 3 & 3 & 2 & 1 \\ \cline{2-5}
           一定期間($T$)& 4 & 2 & 2 & 1 \\ \cline{2-5}
           & 5 & 2 & 2 & 1 \\ \cline{2-5}
           & 6 & 2 & 2 & 1 \\ \cline{2-5}
           & 7 & 1 & 1 & 1 \\ \cline{2-5}
           \hline
        \end{tabular}
      }
\end{table}

\newpage

\subsubsection{前後比較に基づくダイジェスト生成方式の閾値}
「前後比較に基づくダイジェスト生成方式」では式\ref{fig:zengo}において閾値は, 増加率($Th_{inc}$), 減少率($Th_{dec}$), 一定期間($T$)であった. 図\ref{fig:digest2-1}は前後比較に基づくダイジェスト生成方式の抽出例である. 時刻に付いている赤い丸が正解シーンであり, 縦に緑の線がシステムがダイジェストであると判断して選出した箇所である. 図を見ると2箇所正解していることがわかる.

表\ref{tbl:digest2-1}に増加率0.05の時の正解シーン選出の結果を, 表\ref{tbl:digest2-2}に増加率0.10の時の正解シーン選出の結果を示す. 実験の結果, 増加率が0.1, 減少率が0.03, 一定期間が4である時が一番正解数が多く, 適切な値であることがわかった.

\begin{figure}[h]
  \centering
  \includegraphics[width=1\hsize]{fig/digest2-1.eps}
  \caption{前後比較に基づくダイジェスト生成方式の抽出例}
  \label{fig:digest2-1}
\end{figure}

\begin{table}[h]
  \caption{増加率0.05の時の正解シーン選出の結果}
  \label{tbl:digest2-1}
  \centering
      {\small
        \begin{tabular}{|l|l||l|l|l|} \hline
          \multicolumn{5}{|c|}{増加率($Th_{inc}$)0.05} \\ \hline
          & & \multicolumn{3}{c|}{減少率($Th_{dec}$)} \\ \hline
          & & 0.03 & 0.05 & 0.10 \\ \hline \hline \cline{2-5}
          & 1 & 0 & 0 & 0 \\ \cline{2-5}
          & 2 & 1 & 1 & 1 \\ \cline{2-5}
          & 3 & 0 & 1 & 0 \\ \cline{2-5}
          一定期間($T$)& 4 & 0 & 0 & 1 \\ \cline{2-5}
          & 5 & 0 & 0 & 0 \\ \cline{2-5}
          & 6 & 0 & 0 & 0 \\ \cline{2-5}
          & 7 & 0 & 0 & 1 \\ \cline{2-5}
          \hline
        \end{tabular}
      }
\end{table}

\begin{table}[h]
  \caption{増加率0.10の時の正解シーン選出の結果}
  \label{tbl:digest2-2}
  \centering
      {\small
        \begin{tabular}{|l|l||l|l|l|} \hline
          \multicolumn{5}{|c|}{増加率($Th_{inc}$)0.10} \\ \hline
          & & \multicolumn{3}{c|}{減少率($Th_{dec}$)} \\ \hline
          & & 0.03 & 0.05 & 0.10 \\ \hline \hline \cline{2-5}
          & 1 & 0 & 0 & 0 \\ \cline{2-5}
          & 2 & 1 & 1 & 0 \\ \cline{2-5}
          & 3 & 1 & 1 & 0 \\ \cline{2-5}
          一定期間($T$)& 4 & 2 & 1 & 0 \\ \cline{2-5}
          & 5 & 0 & 1 & 1 \\ \cline{2-5}
          & 6 & 0 & 0 & 0 \\ \cline{2-5}
          & 7 & 0 & 0 & 1 \\ \cline{2-5}
          \hline
        \end{tabular}
      }
\end{table}

\newpage

\subsubsection{最小二乗法に基づくダイジェスト生成方式}
「最小二乗法に基づくダイジェスト生成方式」では式\ref{fig:jijo}において閾値は, 角度(度), データ数(個)であった. 図\ref{fig:digest3-1}は最小二乗法に基づくダイジェスト生成方式の抽出例である. 時刻に付いている赤い丸が正解シーンであり, 縦に緑の線がシステムがダイジェストであると判断して選出した箇所である. 図を見ると5箇所正解していることがわかる.

表\ref{tbl:digest3-1}に正解シーン選出の結果を示す. 実験の結果, 角度(度)55, データ数(個)3である時が一番正解数が多く, 適切な値であることがわかった.

\begin{figure}[h]
  \centering
  \includegraphics[width=1\hsize]{fig/digest3-1.eps}
  \caption{最小二乗法に基づくダイジェスト生成方式の抽出例}
  \label{fig:digest3-1}
\end{figure}

\begin{table}[h]
  \caption{正解シーン選出の結果}
  \label{tbl:digest3-1}
  \centering
      {\small
        \begin{tabular}{|l|l||l|l|l|l|l|l|l|l|l|l|l|l|} \hline
          & & \multicolumn{12}{|c|}{角度(度)} \\ \hline
          & & 25 & 30 & 35 & 40 & 45 & 50 & 55 & 60 & 65 & 70 & 75 & 80 \\ \hline \hline
          & 2 & 0 & 0 & 0 & 2 & 3 & 3 & 2 & 1 & 1 & 0 & 0 & 0 \\ \cline{2-14}
          データ数(個) & 1 & 1 & 1 & 1 & 1 & 1 & 3 & 3 & 2 & 2 & 2 & 2 & 2 \\ \cline{2-14}
          & 0 & 0 & 0 & 0 & 0 & 0 & 0 & 2 & 2 & 2 & 2 & 2 & 2 \\ \cline{2-14}
          & 0 & 0 & 0 & 0 & 0 & 0 & 1 & 1 & 1 & 1 & 1 & 1 & 1 \\ \cline{2-14}
          \hline
        \end{tabular}
      }
\end{table}

\subsection{提案方式の評価}
今までの実験により, それぞれの提案方式の適切な閾値がわかった. 次に, その適切な閾値を適応したそれぞれの提案方式を比較し, P2Pライブストリーミングにおいてダイジェストを生成する際に最も適当なダイジェスト生成方式を決定する.

3つの動画に対して, それぞれの提案方式の適合率を計算することによって評価する. 適合率は式\ref{siki:tekigo}に従う. これは, 提案方式がダイジェストとして選出したもののうち, 元々用意しておいた正解シーンが含まれている確立を意味する. 適合率の値が高いほど, より適切なダイジェスト生成方式であると言える. 表\ref{tbl:tekigo}に実験の結果を示す. 実験の結果, 最小二乗法に基づくダイジェスト生成方式の適合率が最も高く, 平均で66\%を超えるという結果になった.

\begin{eqnarray}
  適合率 = \frac{正解シーン数}{方式による抽出シーン数}
  \label{siki:tekigo}
\end{eqnarray}

\begin{table}[h]
  \caption{適合率の結果}
  \label{tbl:tekigo}
  \centering
      {\small
        \begin{tabular}{|l|l|l|l|} \hline
          & \multicolumn{3}{|c|}{適合率} \\ \hline
          生成方式 & 閾値に基づく方式 & 前後比較に基づく方式 & 最小二乗法に基づく方式 \\ \hline
          動画1 & 12.5\% & 50.0\% & 62.5\% \\ \hline
          動画2 & 37.5\% & 50.0\% & 75.0\% \\ \hline
          動画3 & 50.0\% & 50.0\% & 62.5\% \\ \hline
        \end{tabular}
      }
\end{table}

\subsection{ダイジェスト生成方式に対する評価の考察}
ダイジェスト生成方式の実験では, まずそれぞれの提案方式の適切な閾値を決定した. 次にその閾値を適応した提案方式の適合率を計算することによって最も適切なダイジェスト生成方式を決定した.

適切な閾値の決定では, 「閾値に基づくダイジェスト生成方式」, 「前後比較に基づくダイジェスト生成方式」, 「最小二乗法に基づくダイジェスト生成方式」のそれぞれで閾値を決定した. それぞれについて考察していく. 次に適合率の結果についての考察を行う.

\subsubsection{閾値に基づくダイジェスト生成方式}
閾値は増加率0.1, 減少率0.03, 一定時間3という結果であった. 増加率は減少率よりも高い値であった. これは, 瞬間的に多くのユーザがコメントをして, その瞬間に対して反応したことを意味している. 瞬間的に多くの反応があったということは, とても注目度が高いことを意味しており, ダイジェストにすべき瞬間であったことがわかる. まだ, 一定時間は3という結果であり, 長くもなく短くもない値であった. これは, 短すぎると雑音となる部分まで余計に反応してしまうためであり, 逆に長すぎると瞬間的な変化に反応出来ないからであると考えられる.

\subsubsection{前後比較に基づくダイジェスト生成方式}
閾値は増加率0.1, 減少率0.03, 一定時間4という結果であった. この閾値は「閾値に基づくダイジェスト生成方式」の結果とほぼ同じであり, 基本的には同様の考え方が出来る. しかし, こちらの方式では増加率が0.05の場合だとほとんど正解シーンを選出出来なかったことがわかる. これは「前後比較に基づくダイジェスト生成方式」の方では特に増加率の値に敏感であり, 細かい値を設定することが望ましいことがわかった.

\subsubsection{最小二乗法に基づくダイジェスト生成方式}
閾値は角度(度)55, データ数(個)3という結果であった. 全体的に角度は急なほど正解シーンを多く選出していることがわかる. これは, より瞬間的な状態を検出していることがわかる. データ数は少ない方が多く正解シーンを選出していることがわかる. これは, データ数が多いほどグラフをなだらかにしてしまい, 瞬間的な状態を検出出来なかったからではないかと考えられる.

\subsubsection{適合率の結果について}
適合率は「最小二乗法に基づくダイジェスト生成方式」が最も適切であることがわかった. これは, その他の方式では雑音となる突発的な瞬間を誤検知してしまったのに対し, こちらの方式では最小二乗法を適応していることで, より自然な盛り上がり部分をダイジェストとして捉え, 検出出来ているからではないかと考えられる.

\section{トポロジ設計に対する評価}
トポロジ設計に対する評価を行う. 評価はシミュレーションで行った. シミュレーションではNS-2を用いた. 提案システムでは各ノードに役割を与えたトポロジを設計した. その各役割の適切な割合についての評価を行う. また, 役割が適切に機能しているかの評価を行う. さらに, P2Pネットワーク内に十分なダイジェストを保有できているかを評価する.

% ここから下少し適当

\subsection{前提条件}
※ここから少し適当

シミュレーションを行う上でいくつかの前提となる条件について述べる. 前提となる項目は「帯域幅分布」, 「コメント数分布」, 「ノード数とクラスタ」がある. それぞれについての前提条件を決定する.

全ノードには固有の帯域幅を割り当てる. 「帯域幅分布」は表\ref{tbl:band-dist}のように決定する.

\begin{table}[h]
  \caption{帯域幅分布}
  \label{tbl:band-dist}
  \centering
      {\small
        \begin{tabular}{|c|c|c|c|c|c|c|c|c|c|c|} \hline
          \shortstack{帯域幅 \\ (kbps)} & 256 & 320 & 384 & 448 & 512 & 640 & 768 & 1024 & 1500 & 3000 \\ \hline
          \shortstack{割合 \\ (\%)} & 10.0 & 14.3 & 8.6 & 12.5 & 2.2 & 1.4 & 6.6 & 28.1 & 1.4 & 14.9 \\ \hline
        \end{tabular}
      }
\end{table}

また全ノードには固有のコメント数を割り当てる. サンプルとして1つの動画に対して分析を行った結果を利用する\cite{comment}. 平均コメント数は4で標準偏差は6.8である. 分析の結果を図\ref{fig:comment-res}に示す.

\begin{figure}[h]
  \centering
  \includegraphics[width=1\hsize]{fig/comment-res.eps}
  \caption{コメント数分析の結果}
  \label{fig:comment-res}
\end{figure}

固有のコメント数を割り当てた結果である, 「コメント数分布」は表\ref{tb:comment-dist}のように決定する.

\begin{table}[h]
  \caption{コメント数分布}
  \label{tbl:comment-dist}
  \centering
      {\small
        \begin{tabular}{|c|c|c|c|c|c|c|c|c|c|c|} \hline
          \shortstack{コメント数} & 2 & 5 & 7 & 10 & 12 & 15 & 17 & 20 & 22 & 25 \\ \hline
          \shortstack{割合 \\ (\%)} & 16 & 17 & 17 & 16 & 15 & 13 & 10 & 2 & 1 & 1 \\ \hline
        \end{tabular}
      }
\end{table}



全ノード数に応じたクラスタの数を予め設定する. 全ノード数とクラスタ数対応表である, 「ノード数とクラスタ」を表\ref{tbl:cluster-dist}のように決定する.

\begin{table}[h]
  \caption{ノード数とクラスタ数対応表}
  \label{tbl:cluster-dist}
  \centering
      {\small
        \begin{tabular}{|c|c|c|c|c|} \hline
          \shortstack{ノード数} & 200 & 400 & 600 & 800 \\ \hline
          \shortstack{クラスタ数} & 7 & 10 & 14 & 18  \\ \hline
        \end{tabular}
      }
\end{table}

その他の前提条件として, ノード間遅延を全てのノード間で100msに設定している. またパケットロス率は0\%として考慮しないこととする. また, 参加離脱を考慮した実験においては, 10秒に1回参加と離脱を行うこととする. 参加はランダムであり, 最初はダイジェスト未取得のノーマルノードとなる. ダイジェスト未取得のノーマルノードは24秒経つとダイジェスト取得済みのダイジェストノードとなる. また, シミュレーションにおいての参加離脱はノードにパケットが流れたかどうかで判断する.

ノード数が200の時の、役割を与えた場合と与えない場合のトポロジ図をそれぞれ図\ref{fig:topology-role}と図\ref{fig:topology-no-role}に示す.

\newpage

\begin{figure}[h]
  \centering
  \includegraphics[width=1\hsize]{fig/topology-role.eps}
  \caption{役割を与えた場合のトポロジ図}
  \label{fig:topology-role}
\end{figure}

\newpage

\begin{figure}[h]
  \centering
  \includegraphics[width=1\hsize]{fig/topology-no-role.eps}
  \caption{役割を与えない場合のトポロジ図}
  \label{fig:topology-no-role}
\end{figure}

\newpage

\subsection{適切な役割の割合の決定}
提案するトポロジにはいくつか役割が存在する.このうち「ゲートノード」, 「セミゲートノード」, 「ダイジェストノード」の役割の適切な割合を決定する. まず最初に「ダイジェストノード」の割合を, 次に「ゲートノード」と「セミゲートノード」の割合を決定する.

\subsubsection{ダイジェストノードの割合}
ダイジェストノードの割合をそれぞれ, 全ノードの10\%, 20\%, 30\%で割合を変え, スループットの値を比較した. 図\ref{fig:digest-rate}に結果を示す.

\begin{figure}[h]
  \centering
  \includegraphics[width=1\hsize]{fig/digest-rate.eps}
  \caption{ダイジェストノードの割合を変化させた時の結果}
  \label{fig:digest-rate}
\end{figure}

実験の結果, 10\%の時の平均スループットが265.71kbps, 20\%の時の平均スループットが573.12kbps, 30\%の時の平均スループットが414.63\%であった. 20\%の時が最もスループットの値が大きく, 適切な値であることがわかった.

\subsubsection{ゲートノードとセミゲートノードの割合}
次にゲートノードとセミゲートノードの割合を決定する. ゲートノードとセミゲートノードは常に1対2の関係となるため, 一緒に決定する. 全ノードのそれぞれ10\%と20\%, 20\%と40\%で割合を変え, スループットの値を比較した. 図\ref{fig:gate-semi-rate}に結果を示す.

\begin{figure}[h]
  \centering
  \includegraphics[width=1\hsize]{fig/gate-semi-rate.eps}
  \caption{ゲートノードとセミゲートノードの割合を変化させた時の結果}
  \label{fig:gate-semi-rate}
\end{figure}

実験の結果, 10\%と20\%の時の平均スループットが446.33kbps, 20\%と40\%の時の平均スループットが413.48kbpsであった. どちらもそれほど変わらないが, 10\%と20\%の方がスループットの値が大きかった. しかし, グラフを見るとノード数が小さい時は10\%と20\%の時の方が値が高く, ノード数が大きい時は20\%と40\%の時の方が値が高かった. この結果, ノード数によって割合を変えることが良いことがわかった.

\subsection{役割の適切性に対する評価}
次に役割を与えたことの適切性に対する評価を行う. 図\ref{fig:throughput-role}に役割を与えた場合のスループットの推移の様子を示す.

\begin{figure}[h]
  \centering
  \includegraphics[width=1\hsize]{fig/throughput-role.eps}
  \caption{各ノード数におけるスループットの推移}
  \label{fig:throughput-role}
\end{figure}

\newpage

図\ref{fig:throughput-no-role}に役割を与えない場合のスループットの推移の様子を示す.

\begin{figure}[h]
  \centering
  \includegraphics[width=1\hsize]{fig/throughput-no-role.eps}
  \caption{役割を与えない場合の各ノード数におけるスループットの推移}
  \label{fig:throughput-no-role}
\end{figure}

\newpage

図\ref{fig:throughput-compare}に2つの場合を比較した様子を示す.

\begin{figure}[h]
  \centering
  \includegraphics[width=1\hsize]{fig/throughput-compare.eps}
  \caption{役割を与えない場合の各ノード数におけるスループットの推移の平均の比較}
  \label{fig:throughput-compare}
\end{figure}

\newpage

\subsection{ダイジェスト保有率に対する評価}
一定時間経つとダイジェスト未取得ノードはダイジェスト取得ノードになる

ダイジェスト保有率の式

ダイジェスト保有率の結果の表

\subsection{トポロジ設計に対する評価の考察}
役割を与えることの有用性確認できた、接続数を分散できたため

ダイジェスト保有率は高かった、コメント数の多いノードにダイジェストをもたせたため



\clearpage
\chapter{まとめと今後の課題} \label{txt:conclusion}
\section{結論}
本研究では, P2Pライブストリーミングにおいて, 途中参加したユーザがダイジェスト視聴可能なP2Pライブストリーミングシステムを提案した. システムを実現するために2つの段階を考えた. 1つ目はP2Pネットワーク内でダイジェストを生成することで, 2つ目は作成したダイジェストをP2Pネットワーク内で保持し, 広めるためのトポロジ設計を行うことである.

1つ目の段階ではダイジェスト生成方式を提案した. 3つの方式を提案し, それぞれのダイジェストの適合率を比較した結果, 「最小二乗法に基づくダイジェスト生成方式」が最も適切で, 映像全体の66\%以上のダイジェストを判定出来ることが分かった.

2つ目の段階ではトポロジを提案した. 複数クラスタ型において, 各ノードに役割を持たせたトポロジを設計した. シミュレーションでは, 役割を持たせた場合と役割を持たせない場合のネットワーク全体のスループットの値を比較した. その結果, 役割を持たせた場合の平均のスループットの値は406.73kbpsと高く, またノード数が大きくなってもスループットの値は下がらなかった. 役割を持たせることの有用性を確認することが出来た.

\section{今後の課題}
ダイジェスト生成方式においては, どのような種類にも対応可能な汎用的なダイジェスト生成方式を目指していた. しかし, 本研究では3種類の動画を対象とした実験しか出来なかった. 今後はより多くの種類の動画に対して方式を適用して有用性の確認をしていく予定である.

また, トポロジ設計では既存のP2Pライブストリーミングシステムとは定性的な評価しか行えなかった. 同じ条件下でシミュレーションを行い, 他研究との比較を行うべきである. さらに, 本研究ではシミュレーション上での評価にとどまっている. 実際のP2Pネットワーク上でシステムを動かし, 有用性の検証をすべきである.



\clearpage

\ackn{
本研究を遂行するにあたり、様々な方々にお世話になりました。

指導教員である末田欣子先生には, 仕事が忙しいにも関わらず毎週行われたゼミでの終始熱心なご指導を頂きました. また, ゼミ以外でもメール等を介して常に適切な助言を賜りました. ここに厚く御礼申し上げます.

講座内進捗や講座内輪講などにおいて研究を進める上での貴重な意見や助言を頂いた, 多田好克先生, 小宮常康先生, 鶴岡行雄先生, 本庄利守先生には深く感謝します.

研究室の先輩である中原祥吾様, 他の研究室の先輩である藤田竜一様, 村井栄王様, 森中翔太郎様, 片桐国建様, 石田峰文様, 神保直幸様, 若井英之様, 講座の同期である熊谷佑弥様, 山本峻丸様, 吉原大夢様, 磯谷俊明様, 工藤朋哉様, 講座の後輩であるグエン クアン ヒエップ様, 木田純平様, 関湧大様, ゾリーグ ウンダラム様, 李明元様, 新タ智啓様とは日々の研究室生活を共に過ごさせて頂き, 多くの助言を頂きました. ここに感謝の意を表します.

さらに, 私が電気通信大学大学院で様々なことを学ぶことが出来たことは, 両親, 友人, 先生, 先輩, 後輩など, これまでの人生で出会った方々のおかげです. 皆様への心から感謝の気持ちと御礼を申し上げたく、謝辞にかえさせていただきます。

\begin{flushright}
  2015年1月吉日 鈴木駿介
\end{flushright}
}

\clearpage

%
%       ・著者が2人の場合は A and B、複数の場合は A, B, C, and D と書く
%       ・著者名と論文タイトルは : で区切る
%       ・論文タイトルは`` '' で囲む
%       ・論文誌名は斜字体で書く
%       ・図書を参照する場合は ISBN 番号も書く
%       ・URL はなるべく参考文献に入れない(資料がそこでしか手に入らない
%         場合はやむを得ないが)
%
\begin{thebibliography}{99}
%\addcontentsline{doc}{section}{参考文献}

\bibitem{nico}
  ニコニコ生放送, http://live.nicovideo.jp/, 2015年閲覧.
\bibitem{twi}
  Twicasting, http://twitcasting.tv/, 2015年閲覧.
\bibitem{ust}
  Ustream, http://www.ustream.tv/, 2015年閲覧.
\bibitem{afr}
  AfreecaTV, http://www.afreeca.com/, 2015年閲覧.
\bibitem{hcps}
  Yang Guo, Chao Liang, and Yong Liu, “Hierarchically Clustered P2P Video Streaming: Design, implementation, and evaluation,” Computer Networks, pp.3432-3445, 2012.
\bibitem{chojo}
  元橋 智紀, 藤本 章宏, 廣田 悠介, 戸出 英樹, 村上 孝三, “多様な配信木により離脱耐性と遅延抑制を向上させる重畳クラスタ木型動画配信システム,” 通信技術の革新を担う学生論文特集, p.132-142, 2014年.
\bibitem{metree}
  Huey-Ing Liu and I-Feng Wu, “MeTree: A Contribution and Locality-Aware P2P Live Streaming Architecture,” AINA 24th IEEE International Conference on, pp.1136-1143, 2010.
\bibitem{dis}
  V. Padmanabhan, H. Wang, P. Chou, and K. Sripanidkulchai, “Distributing streaming media content using cooperative networking,” Network and Operating Systems Support for Digital Audio and Video, pp.177–186, May 2002.
\bibitem{streamline}
  G. Bianchi, N. Melazzi, L. Bracciale, F. Piccolo, and S. Salsano, “Streamline: An optimal distribution al-gorithm for peer-to-peer real-time streaming,” IEEE Trans. Parallel Distrib. Syst., vol.21, no.6, pp.857–871, June 2010.
\bibitem{pdms}
  橋本 隆子, 加登 岡隆, 飯沢 篤志, ”スポーツ映像におけるシーン重要度算出アルゴリズムとその評価,” データ工学ワークショップ, 2003.
\bibitem{yakyu}
  熊野 雅仁, 有木 康雄, 塚田 清志, “野球中継のハイライトシーン実時間配信を目的とした特徴のマイニングによるPCシーンの自動検出,” 映像情報メディア学会誌, vol.59, No.1, pp.77-84, 2005.
\bibitem{band-dist}
  Zhengye Liu, Yanming Shen, Ross K.W., Panwar S.S. and Yao Wang, “Substream Trading: Towards an open P2P live streaming system,” International Conference on Network Protocols, pp.94-130, 2008.
\bibitem{band-dist-1}
  C. Huang, J. Li, and K. W. Ross, "Can Internet VoD be profitable?," ACM SIGCOMM, 2007.
\bibitem{band-dist-2}
  M. Dischinger, A. Haeberlen, K. P. Gummadi, and S. Saroiu, "Characterizing residential broadband networks," ACM IMC, 2007.
\bibitem{cluster-dist}
  大村淳己,高田和也,後藤滋樹, “Location Based Clusteringを用いたP2Pストリーミング,電子情報通信学会技術研究報告,” Vol. 110, No. 373, IN2010-124, pp.37-42, 2011.
\bibitem{comment}
  “ニコニコ動画のコメントを分析してみる@2013年冬アニメ”, \\ http://blog.livedoor.jp/mgpn/archives/51887779.html, 2014年閲覧.
\end{thebibliography}


\appendix
\chapter{シミュレーションで使用したコード}

\section{役割有無実験}

\subsection{役割有りメインコード}
\label{txt:not-join-leave-my-goddard}
\lstinputlisting[
  language=tcl,
  escapechar={},
  basicstyle=\tiny,
  numbers=left,
  breaklines=true,
  caption=役割有りメインコード,
  label=code:not-join-leave-my-goddard
]
{code/not-join-leave/my-goddard.tcl}

\subsection{役割無しメインコード}
\label{txt:not-join-leave-my-goddard-no-roll}
\lstinputlisting[
  language=tcl,
  escapechar={},
  basicstyle=\tiny,
  numbers=left,
  breaklines=true,
  caption=役割無しメインコード,
  label=code:not-join-leave-my-goddard-no-roll
]
{code/not-join-leave/my-goddard-no-roll.tcl}

\subsection{外部プロシージャ共通コード}
\label{txt:not-join-leavemy-goddard-procs}
\lstinputlisting[
  language=tcl,
  escapechar={},
  basicstyle=\tiny,
  numbers=left,
  breaklines=true,
  caption=外部プロシージャ共通コード,
  label=code:not-join-leave-my-goddard-procs
]
{code/not-join-leave/my-goddard-procs.tcl}

\subsection{デフォルトパラメータコード}
\label{txt:not-join-leave-my-goddard-default}
\lstinputlisting[
  language=tcl,
  escapechar={},
  basicstyle=\tiny,
  numbers=left,
  breaklines=true,
  caption=デフォルトパラメータコード,
  label=code:not-join-leave-my-goddard-default
]
{code/not-join-leave/my-goddard-default.tcl}

\section{参加離脱実験}

\subsection{メインコード}
\label{txt:join-leave-low-data-my-goddard}
\lstinputlisting[
  language=tcl,
  escapechar={},
  basicstyle=\tiny,
  numbers=left,
  breaklines=true,
  caption=メインコード,
  label=code:join-leave-low-data-my-goddard
]
{code/join-leave-low-data/my-goddard.tcl}

\subsection{外部プロシージャコード}
\label{txt:join-leave-low-data-my-goddard-procs}
\lstinputlisting[
  language=tcl,
  escapechar={},
  basicstyle=\tiny,
  numbers=left,
  breaklines=true,
  caption=外部プロシージャコード,
  label=code:join-leave-low-data-my-goddard-procs
]
{code/join-leave-low-data/my-goddard-procs.tcl}

\subsection{デフォルトパラメータコード}
\label{txt:join-leave-low-data-my-goddard-default}
\lstinputlisting[
  language=tcl,
  escapechar={},
  basicstyle=\tiny,
  numbers=left,
  breaklines=true,
  caption=デフォルトパラメータコード,
  label=code:join-leave-low-data-my-goddard-default
]
{code/join-leave-low-data/my-goddard-default.tcl}



\end{document}
