\documentclass[a4paper,12pt]{jsbook}
\usepackage{fancyhdr} %デザイン設定
\usepackage{shuron} %修論用スタイル
\usepackage{url} %URL表示用
\usepackage[dvipdfmx]{graphicx} %画像ファイル用
\usepackage{flafter} %図の位置が図参照より後になるように
\usepackage{subfigure} %部分図
\usepackage{mediabb} %PDFファイルを画像として参照
\usepackage{listings,jlisting} %プログラムコード用
\usepackage{amsmath, amssymb} %オーダ記号などの数学記号
%\usepackage{theorem} %定義
\usepackage{multirow} %縦に跨がる表
\usepackage{multicol} %段落
\usepackage{algorithm}
%\usepackage{algorithmic}
\usepackage{algcompatible}
\usepackage[dvipdfmx]{hyperref}
\usepackage{pxjahyper}

%% \renewcommand{\lstlistingname}{プログラム}
%% \renewcommand{\lstlistlistingname}{プログラムコード一覧}
\lstset{language=Java,
        basicstyle=\footnotesize,
        commentstyle=\textit,
        classoffset=1,
        keywordstyle=\bfseries,
         showstringspaces=false,
        tabsize=4,
        frame=lines,
        numbers=left,
        numberstyle=\scriptsize,
        numbersep=5pt,
        escapechar=\#
}
%\theoremstyle{break}
%\newtheorem{define}{定義}
%\newcommand{\transprog}{$ P \xrightarrow{\mathcal{T}} P'$}

\newcommand{\textref}[1]{\ref{#1}節(p.\pageref{#1})}
\newcommand{\figref}[1]{\figurename~\ref{#1}}
\newcommand{\tblref}[1]{\tablename~\ref{#1}}
\newcommand{\cdref}[1]{\lstlistingname~\ref{#1}}
\newcommand{\chapterref}[1]{\prechaptername \ref{#1} \postchaptername}

\def\PT{$\mathsf{P^T}$}
\def\PF{$\mathsf{P^F}$}

\makeatletter
  \def\@cite#1{\textsuperscript{[#1]}}
  \renewcommand{\ALG@name}{変換規則}
\makeatother

\voffset=-10.5mm

\begin{document}
\setcounter{page}{0}
\thispagestyle{empty}
\noindent
\begin{tabular}{c}
{\ueclogo B} \vspace{1.5cm}     \\
{\Large 平成26年度 修士論文}    \\
\end{tabular}

\vspace{2.0cm}

\begin{center}
{\huge \bf ロールベースのP2Pライブ \\ ストリーミングアーキテクチャの研究}
\end{center}

\vspace{3cm}

\LARGE
\begin{flushright}
電気通信大学~大学院情報システム学研究科         \\
情報システム基盤学専攻                          \\
1353015~~鈴木 駿介                              \\

\vspace{2cm}

{\def\arraystretch{0.6}
\begin{tabular}{rll@{}}
指導教員        & 多田 好克     & 教授                \\
                & 末田 欣子     & 客員准教授                  \\
                & 古賀 久志     & 准教授                \\
                                                        \\
提出日          & \multicolumn{2}{c@{}}{平成27年1月26日}        \\
\end{tabular}
}
\end{flushright}

\normalsize
\newpage

\setcounter{tocdepth}{2}
\pagenumbering{roman}
\tableofcontents
%\setcounter{page}{1}

\listoffigures

\listoftables

%% \lstlistoflistings

%%%%%%%%%%%%%%%%%%%%%%%%%%%%%%%%%%%%%%%%%%%%%%%%%%%%%%%%%%%%%%%%%%%%%%
\chapter{導入}
\pagenumbering{arabic}
ネットワーク技術の発展に伴い, 近年ではリアルタイム動画配信サービスが人気となっている. このようなサービスはサーバから各クライアントに配信する, サーバ-クライアント方式が一般的であるが, 配信サーバのコスト削減などの理由で, サーバを介さずに直接クライアント同士で配信を行うP2Pを利用したライブストリーミング配信が期待されている. しかし, 既存のP2Pライブストリーミングシステムでは, 途中参加したユーザはそれまでの配信内容を把握することが出来ないといった問題がある. そこで本研究ではP2Pライブストリーミングにおいて, 途中参加したユーザがダイジェスト視聴可能なP2Pライブストリーミングシステムを提案する. 本研究では2つの段階を考えた. 1つ目はP2Pネットワーク内でダイジェストを生成することである. 2つ目は作成したダイジェストをP2Pネットワーク内で保持し, 広めるためのトポロジ設計を行うことである.

1つ目の段階ではダイジェスト生成方式を提案した. 具体的には「増減率に基づくダイジェスト生成方式」, 「前後比較に基づくダイジェスト生成方式」, 「最小二乗法に基づくダイジェスト生成方式」の3つを提案した. 提案するシステムは参加ユーザが自由にコメント投稿出来ることを想定し, 3つの方式ともコメントをした参加ユーザ数を利用した. 「増減率に基づくダイジェスト生成方式」ではユーザ数が増加している時を始点, 減少している時を終点とし, 始点から終点で最もユーザ数の多い点をダイジェストとした. 「前後比較に基づくダイジェスト生成方式」ではある点についてユーザ数が前後で急激に増減している点をダイジェストとした. 「最小二乗法に基づくダイジェスト生成方式」では最小二乗法を適用して2直線の重なりが鋭角な点をダイジェストとした. 実験では, ある動画に対して予め正解シーンを用意し, 提案方式がダイジェストとして選出したシーンとその正解シーンとの適合率で評価を行った. その結果「最小二乗法に基づくダイジェスト生成方式」が最も適合率が高く, 適切な方式であることがわかった.

2つ目の段階ではトポロジを提案した. 具体的には複数クラスタ型において, 各ノードに役割をもたせたトポロジを設計した. 役割として「ゲートノード」, 「セミゲートノード」, 「ダイジェストノード」を用意した. ダイジェスト生成方式ではコメントをしたユーザ数を利用したが, トポロジ設計では各ノードの役割を決定するのに各コメント数におけるユーザ数の割合を利用した. ゲートノードは配信者ノードから配信されたコンテンツのパケットをクラスタ内で一番最初に取得する役目を担う. 高帯域かつコメント数の多いノードを選出した. セミゲートノードはクラスタ内でゲートノードから受信したパケットをクラスタ内部に拡散する役目を担う. ゲートノードの次に高帯域でコメント数の多いノードを選出した. ダイジェストノードはダイジェストの作成, また作成したダイジェストを保有し, 新規参加ノードへ送信する役目を担う. 最もコメント数の多いノードを選出した. 実験ではNS-2というシミュレーションを使って評価を行った. 役割の適切性に対する評価では, 役割を与えない場合ではノード数が増えるに従ってスループットの値が下がっていくが, 役割を与えた場合ではスループットの値が下がること無く, ノード数が増えても継続的に高いスループットを示しており, 役割を与えることの有用性を確認することが出来た.


\clearpage
\chapter{背景} \label{txt:introductions}
ネットワーク技術の発達に伴い, ネットワークを用いた動画配信サービスが人気である. 特にリアルタイム動画配信サービスが人気であり, 日本ではニコニコ生放送\cite{nico}やTwitCasting\cite{twi}, アメリカではUstream\cite{ust}, 韓国ではAfeecaTV\cite{afr}と, 近年世界中で急速に普及している.  このようなサービスの形態としては, サーバから配信された映像をクライアントが視聴する, サーバ-クライアント方式が一般的である. そこで, 配信サーバのコスト削減や配信者の負荷を軽減するため, サーバを介さず直接クライアント同士で配信を行うP2P(peer to peer)を利用したライブストリーミング配信が期待されている. しかし, 既存のP2Pライブストリーミングシステムでは, 途中参加したユーザはそれまでの配信内容を把握することが出来ないといった問題がある. そこで本研究ではP2Pライブストリーミングにおいて, 途中参加したユーザがダイジェスト視聴可能なP2Pライブストリーミングシステムを提案する.

本研究では2つの段階を考えた. 1つ目はP2Pネットワーク内でダイジェストを生成することである. 2つ目は作成したダイジェストをP2Pネットワーク内で保持し, 広めるためのトポロジ設計を行うことである. まずはじめにダイジェスト生成方式について述べ, その次にトポロジ設計について述べる.

\newpage

\section{ダイジェスト生成方式}
動画配信サービスにおいてダイジェストを見ることは, 動画全体の雰囲気を知るために有用である. リアルタイム動画配信であるライブストリーミングにおいても, ユーザが途中から配信に参加した場合にそれまでの配信の内容を把握出来るという点においてダイジェストを見ることは有用である. しかし, P2Pのライブストリーミングにおいては, サーバ-クライアント方式と比べてダイジェストを保存しておくサーバを用意することが出来ないといった問題があり, 既存のサーバ-クライアント方式のためのダイジェスト生成方式が適応出来ない. 以下にダイジェスト生成方式の既存研究をあげる.

\subsection{動画に対するダイジェスト生成方式}
橋本らはスポーツ映像を対象として, 映像メタデータと利用者の嗜好情報を利用したパーソナルダイジェスト生成方式PDMS(Personal Digest Making Scheme)\cite{pdms}を提案している. PDMSは, 映像メタデータから発生事象の重要度を自動的に検出し, 複数のダイジェストを選択する. 野球の映像を対象とし, 事前に選んだ正解集合と比較して適合率によってダイジェスト配信システムの重要度算出アルゴリズムの評価を行った.

PDMSでは一度映像を整理してシーン毎に分類し, その上でダイジェストシーンを選出している. そのためライブ映像に対してはPDMSの手法を適応することが出来ない.

\subsection{ライブ映像に対するダイジェスト生成方式}
熊野らは野球の実況中継映像を対象として, 自動的にインデックス情報を付与して, ハイライトシーンを検出する, リアルタイムダイジェスト生成システム\cite{yakyu}を提案している. システムでは特に野球のPC(Picher and Catcher)シーンを画像解析により検出し, さらに音声解析により特別なイベントと判断されたキーワードを含む区間をハイライトシーンとして生成している. 実験では予め正解であるシーンを用意し, システムを適応した際の適合率により評価を行った. 結果は最も高い適合率で97.2\%という結果であり, 有用であることを示している.

熊野らの研究ではライブ映像に対してダイジェストをリアルタイムに生成している点が優れている. しかし, 野球という特定の分野の映像を対象にしているため, 様々な内容の動画配信に対応させることは難しい.

\newpage

\section{P2Pライブストリーミングのトポロジ}
P2Pライブストリーミングは主にアプリケーションレベルマルチキャスト(ALM: Application Level Multicast)によって行われる. ALMはアプリケーションによって実現されるため開発が容易で多くの研究がなされている.

ALMは主にツリー型とメッシュ型に分類される. ツリー型は遅延が少なく構築が容易であるというメリットがあるが, 耐故障性やノードの離脱に弱いといったデメリットがある. メッシュ型は耐故障性やノードの離脱に強いというメリットがあるが, 複数の経路を用いるため遅延が大きくなるといったデメリットがある. 一方でこれらツリー型やメッシュ型の欠点を補うために複数クラスタ型\cite{dis},\cite{streamline}が提案されている. 複数クラスタ型は複数のクラスタを構築し各サブストリームをそれぞれのクラスタで配信するといった方法である. 複数クラスタ型ではツリー構造のように中継ノードにストリームを渡すため, 配信の負荷が軽減される. またクラスタ内部では各ノードが複数の経路を持つため耐故障性に優れる. 次ページ以降に複数クラスタ型の既存研究をあげる.

\newpage

\subsection{階層型クラスタ構造}
Yang Guoらは階層的クラスタHCPS(Hierarchically Clustered P2P Video Streaming)を提案している\cite{hcps}. 図\ref{fig:hcps}は階層型クラスタHCPSのトポロジ設計である. HCPSはクラスタ間で帯域のバランスを取ることにより, より品質の高い映像を流すことを可能にしている. また, 隣のピアへのストリームの受け渡しの際に, アップロード帯域幅を有効活用出来るようなスケジューリングアルゴリズムを提案している. クラスタ内部は完全結合となっており, 各クラスタのアップロード容量が均一になるように構成されている. クラスタを形成する際にクラスタ内で一番最初に配信内容を受け取るノードをヘッドノードと定義し, そのアップロード容量の大きいものを選出している.

HCPSはクラスタの中身が完全結合であり, 各ノードへ多くの経路を辿ることになるため遅延が多くなってしまうという課題がある.

\begin{figure}
  \centering
  \includegraphics[width=1\hsize]{fig/hcps.eps}
  \caption{階層型クラスタHCPSのトポロジ設計}
  \label{fig:hcps}
\end{figure}

\subsection{重畳クラスタ木型動画配信システム}
元橋らは重畳クラスタ木方式の動画配信システムを提案している\cite{chojo}. 図\ref{fig:chojo}は重畳型クラスタ木型動画配信システムのトポロジ設計である. 階層的なクラスタ構造になっており, クラスタ間の配信木の中継ノードであるゲートノードが存在する. さらにゲートノードはクラスタ内で一番最初に配信内容を受け取り, それをクラスタ内に広める役割を担っている. ゲートノード選出方式として, P2Pシステム内での滞在時間の長さとRTT(Round Trip Time)を考慮した方法を提案している. 離脱耐性向上型, 配信遅延抑制型, ハイブリッド型の実験をしたところ, ハイブリッド型が最も性能が良いという結果が出ている.

重畳クラスタ木方式は全体としてツリー構造のため下位のクラスタほど配信者からのホップ数が大きくなってしまうという課題がある.

\begin{figure}
  \centering
  \includegraphics[width=1\hsize]{fig/chojo.eps}
  \caption{重畳型クラスタ木型動画配信システムのトポロジ設計}
  \label{fig:chojo}
\end{figure}

\newpage

\subsection{ハイブリッドアーキテクチャ構造}
Huey-Ing Liuらは局所性と貢献度を考慮したハイブリッドアーキテクチャであるMeTreeを提案している\cite{metree}. 図\ref{fig:metree}はMetreeのトポロジ設計である. MeTreeはISP(Internet Service Provider)ごとにクラスタリングを行いクラスタ内のそれぞれのノードをメッシュで接続し, 生成された各クラスタ同士をツリー構造で接続している. ツリー構造とメッシュ構造のハイブリッド構造となっている. 配信内容をより多くのノードに広めるような貢献度の高いピアには質の高い映像を配信し, 逆に貢献度の低いピアには質の低い映像を配信する. 異なる貢献度を持つピアに異なるQoE(Quality of Experience)を与えている. 物理トポロジとオーバーレイを構築するためのピアの貢献度の両方を考慮することにより, 遅延を減少させている.

MeTreeでは貢献度を意識した設計のため貢献度の低いノードは良い映像が見られず, ネットワーク全体のQoEは低下してしまうという課題がある.

\begin{figure}
  \centering
  \includegraphics[width=1\hsize]{fig/metree.eps}
  \caption{Metreeのトポロジ設計}
  \label{fig:metree}
\end{figure}


%% \clearpage
%% \chapter{関連研究} \label{txt:related}
%% \input{tex/related.tex}

\clearpage
\chapter{提案手法} \label{txt:proposed}
ここに提案手法を書く、最初に研究の目的を示す

まず最初にダイジェスト生成方式、次にトポロジ設計
\section{ダイジェスト生成方式}
ダイジェスト生成方式について述べる

3つ提案した

いづれも突出点を発見する手法

\subsection{ダイジェスト生成方式に対する要求条件}
要求条件を提示する

\subsection{閾値に基づくダイジェスト生成方式}
具体的な方法

式

グラフ

\subsection{前後比較に基づくダイジェスト生成方式}
具体的な方法

式

グラフ


\subsection{最小二乗法に基づくダイジェスト生成方式}
具体的な方法

式

グラフ

\subsection{既存のダイジェスト生成方式との比較}

既存手法との定性評価

\section{トポロジ設計}
ダイジェストを生成し、保持し、広めるためのトポロジ設計

\subsection{トポロジ設計に対する要求条件}
要求条件を提示する

\subsection{配信内容に対するコメント}
コメントの役割を書く

\subsection{ノードの役割}
\subsubsection{ゲートノード}
ゲートノードについて

\subsubsection{セミゲートノード}
セミゲートノードについて

\subsubsection{ダイジェストノード}
ダイジェストノードについて

\subsubsection{その他のノード}
トラッカーサーバの存在とノーマルノードについて

\subsection{ノードの役割決定方法}
各役割についての決定方法を書く

役割決定の図

\subsection{ノード間接続方法}

図を説明しながらノード間の接続方法について書く

クラスタ間の図

\subsection{新規参加ピアについて}
新規参加ピアの行動について書く

新規参加ピアの行動のシーケンス図

\subsection{再構築のタイミング}
再構築のタイミング3パータンを書く

場合によっては図を用いる

\subsection{既存のトポロジ設計との比較}

既存手法との定性評価




\clearpage
\chapter{評価} \label{txt:eval}
本研究では,  配信に途中参加したユーザがダイジェスト視聴可能なP2Pライブストリーミングシステムを提案してきた. そのために2つの段階を考え, 1つ目の段階では, P2Pネットワーク内で特別なサーバに頼らない, 汎用的なダイジェスト生成方式を提案した. 2つ目の段階では, 1つ目の段階において生成されたダイジェストをP2Pネットワーク内で保持し, それを拡散させるためのトポロジ設計を提案した.

この章では, まず1段階目のダイジェスト生成方式に対する評価を行い, その次に2段階目のトポロジ設計に対する評価を行う.

ダイジェスト生成方式では, まず3つの提案方式それぞれに設定されている閾値の適切な値を決めるための評価を行う. その次に実際の動画コンテンツを対象に提案システムを適用し, ダイジェスト生成に最も適切な方式を決定する. トポロジ設計では, まず各役割ノードの適切な割合を決めるための評価を行う. その次にその役割ノードが適切な役目を果たしているかの評価を行う. さらに生成されたダイジェストが適切にP2Pネットワーク内で保持され, 視聴出来たかの評価を行う. なお, 評価はシミュレーションで行う.

表\ref{tbl:env}に評価実験を行った環境を示す.

\begin{table}[h]
  \caption{実行環境}
  \label{tbl:env}
  \centering
      {\small
        \begin{tabular}{|l|l|l|} \hline
          環境名 & 規格 & バージョン \\ \hline \hline
          OS & Ubuntu & 12.04 64-bit  \\ \hline
          CPU & Intel Core i7 2.10~GHz & \\ \hline
          メモリ & 8~GB & \\ \hline
          シミュレータ & NS2 & 2.35 \\ \hline
          言語 & OTcl & 1.14 \\ \cline{2-3}
           & TK & 8.5.10 \\ \hline
          可視化ツール & nam & 1.15 \\ \hline
        \end{tabular}
      }
\end{table}

\newpage

\section{ダイジェスト生成方式に対する評価}\label{sec:ev-digest}
\subsection{評価の準備}
ダイジェスト生成方式に対する評価を行う際の準備について述べる. 評価の1つ目は, 3つの提案方式の適切な閾値を決めることである. 2つ目は3つの提案方式のうち最も適切な方式を決定することである.

2つの評価に共通するものとして, 対象となる映像がある. まず配信時間の長さの異なる3つ映像に対して, 時間の10分の1となる1つ2分間の正解シーンを用意した. つまり, 例えば100分の映像を扱う際には5個の正解シーンが存在するということである.

\subsection{適切な閾値の決定}
3つの提案方式の適切な閾値を決定するための評価方法について述べる. まず, 提案システムがダイジェストであると選出したシーンの数が, 元々用意しておいた正解シーンよりも半分以上であるものを集計する. なお, 正解シーンの前後2分間をダイジェストと判断した場合を選出出来たと判断して集計する. 集計した後, 最も多く正解シーンを選出した閾値を適切な値とする. 例えば100分の映像ならば, 正解シーンが5つ存在する. そのうち提案システムが半分以上選出出来たらならばプラス1カウントする. これを3つの映像それぞれについて行い, 最も多く選出出来た(映像は3種類なので最大で3)閾値を適切な値とする.

\subsubsection{増減率に基づくダイジェスト生成方式の閾値}
「増減率に基づくダイジェスト生成方式」では式\ref{fig:sikiiti}において閾値は, 増加率($Th_{inc}$), 減少率($Th_{dec}$)で, パラメータは一定期間($T$)であった. 図\ref{fig:digest1-1}は増減率に基づくダイジェスト生成方式の抽出例である. 時刻に付いている赤い丸が正解シーンであり, 縦に緑の線がシステムがダイジェストであると判断して選出した箇所である. 図を見るとAと記した1箇所で正解していることがわかる.

表\ref{tbl:digest1-1}に増加率0.05の時の正解シーン選出の結果を, 表\ref{tbl:digest1-2}に増加率0.10の時の正解シーン選出の結果を示す. 実験の結果, 増加率が0.1, 減少率が0.03, 一定期間が3である時が一番正解数が多く, 適切な値であることがわかった.

\begin{figure}[h]
  \centering
  \includegraphics[width=1\hsize]{fig/digest1-1.eps}
  \caption{増減率に基づくダイジェスト生成方式の抽出例}
  \label{fig:digest1-1}
\end{figure}

\begin{table}[h]
  \caption{増加率0.05の時の正解シーン選出の結果}
  \label{tbl:digest1-1}
  \centering
      {\small
        \begin{tabular}{|l|l||l|l|l|} \hline
           \multicolumn{5}{|c|}{増加率($Th_{inc}$)0.05} \\ \hline
           & & \multicolumn{3}{c|}{減少率($Th_{dec}$)} \\ \hline
           & & 0.03 & 0.05 & 0.10 \\ \hline \hline \cline{2-5}
           & 1 & 0 & 1 & 1 \\ \cline{2-5}
           & 2 & 0 & 0 & 2 \\ \cline{2-5}
           & 3 & 2 & 2 & 1 \\ \cline{2-5}
           一定期間($T$)& 4 & 1 & 1 & 1 \\ \cline{2-5}
           & 5 & 2 & 2 & 1 \\ \cline{2-5}
           & 6 & 2 & 2 & 1 \\ \cline{2-5}
           & 7 & 1 & 1 & 1 \\ \cline{2-5}
           \hline
        \end{tabular}
      }
\end{table}

\newpage

\begin{table}[h]
  \caption{増加率0.10の時の正解シーン選出の結果}
  \label{tbl:digest1-2}
  \centering
      {\small
        \begin{tabular}{|l|l||l|l|l|} \hline
           \multicolumn{5}{|c|}{増加率($Th_{inc}$)0.10} \\ \hline
           & & \multicolumn{3}{c|}{減少率($Th_{dec}$)} \\ \hline
           & & 0.03 & 0.05 & 0.10 \\ \hline \hline \cline{2-5}
           & 1 & 1 & 1 & 0 \\ \cline{2-5}
           & 2 & 2 & 2 & 1 \\ \cline{2-5}
           & 3 & 3 & 2 & 1 \\ \cline{2-5}
           一定期間($T$)& 4 & 2 & 2 & 1 \\ \cline{2-5}
           & 5 & 2 & 2 & 1 \\ \cline{2-5}
           & 6 & 2 & 2 & 1 \\ \cline{2-5}
           & 7 & 1 & 1 & 1 \\ \cline{2-5}
           \hline
        \end{tabular}
      }
\end{table}

\newpage

\subsubsection{前後比較に基づくダイジェスト生成方式の閾値}
「前後比較に基づくダイジェスト生成方式」では式\ref{fig:zengo}において閾値は, 増加率($Th_{inc}$), 減少率($Th_{dec}$)で, パラメータは一定期間($T$)であった. 図\ref{fig:digest2-1}は前後比較に基づくダイジェスト生成方式の抽出例である. 時刻に付いている赤い丸が正解シーンであり, 縦に緑の線がシステムがダイジェストであると判断して選出した箇所である. 図を見るとAと記した1箇所で正解していることがわかる.

表\ref{tbl:digest2-1}に増加率0.05の時の正解シーン選出の結果を, 表\ref{tbl:digest2-2}に増加率0.10の時の正解シーン選出の結果を示す. 実験の結果, 増加率が0.1, 減少率が0.03, 一定期間が4である時が一番正解数が多く, 適切な値であることがわかった.

\begin{figure}[h]
  \centering
  \includegraphics[width=1\hsize]{fig/digest2-1.eps}
  \caption{前後比較に基づくダイジェスト生成方式の抽出例}
  \label{fig:digest2-1}
\end{figure}

\begin{table}[h]
  \caption{増加率0.05の時の正解シーン選出の結果}
  \label{tbl:digest2-1}
  \centering
      {\small
        \begin{tabular}{|l|l||l|l|l|} \hline
          \multicolumn{5}{|c|}{増加率($Th_{inc}$)0.05} \\ \hline
          & & \multicolumn{3}{c|}{減少率($Th_{dec}$)} \\ \hline
          & & 0.03 & 0.05 & 0.10 \\ \hline \hline \cline{2-5}
          & 1 & 0 & 0 & 0 \\ \cline{2-5}
          & 2 & 1 & 1 & 1 \\ \cline{2-5}
          & 3 & 0 & 1 & 0 \\ \cline{2-5}
          一定期間($T$)& 4 & 0 & 0 & 1 \\ \cline{2-5}
          & 5 & 0 & 0 & 0 \\ \cline{2-5}
          & 6 & 0 & 0 & 0 \\ \cline{2-5}
          & 7 & 0 & 0 & 1 \\ \cline{2-5}
          \hline
        \end{tabular}
      }
\end{table}

\begin{table}[h]
  \caption{増加率0.10の時の正解シーン選出の結果}
  \label{tbl:digest2-2}
  \centering
      {\small
        \begin{tabular}{|l|l||l|l|l|} \hline
          \multicolumn{5}{|c|}{増加率($Th_{inc}$)0.10} \\ \hline
          & & \multicolumn{3}{c|}{減少率($Th_{dec}$)} \\ \hline
          & & 0.03 & 0.05 & 0.10 \\ \hline \hline \cline{2-5}
          & 1 & 0 & 0 & 0 \\ \cline{2-5}
          & 2 & 1 & 1 & 0 \\ \cline{2-5}
          & 3 & 1 & 1 & 0 \\ \cline{2-5}
          一定期間($T$)& 4 & 2 & 1 & 0 \\ \cline{2-5}
          & 5 & 0 & 1 & 1 \\ \cline{2-5}
          & 6 & 0 & 0 & 0 \\ \cline{2-5}
          & 7 & 0 & 0 & 1 \\ \cline{2-5}
          \hline
        \end{tabular}
      }
\end{table}

\newpage

\subsubsection{最小二乗法に基づくダイジェスト生成方式}
「最小二乗法に基づくダイジェスト生成方式」では式\ref{fig:jijo}においてパラメータは, 角度(度), データ数(個)であった. 図\ref{fig:digest3-1}は最小二乗法に基づくダイジェスト生成方式の抽出例である. 時刻に付いている赤い丸が正解シーンであり, 縦に緑の線がシステムがダイジェストであると判断して選出した箇所である. 図を見るとA, B, C, D, Eと記した5箇所で正解していることがわかる.

表\ref{tbl:digest3-1}に正解シーン選出の結果を示す. 実験の結果, 角度(度)55, データ数(個)3である時が一番正解数が多く, 適切な値であることがわかった.

\begin{figure}[h]
  \centering
  \includegraphics[width=1\hsize]{fig/digest3-1.eps}
  \caption{最小二乗法に基づくダイジェスト生成方式の抽出例}
  \label{fig:digest3-1}
\end{figure}

\begin{table}[h]
  \caption{正解シーン選出の結果}
  \label{tbl:digest3-1}
  \centering
      {\small
        \begin{tabular}{|l|l||l|l|l|l|l|l|l|l|l|l|l|l|} \hline
          & & \multicolumn{12}{|c|}{角度(度)} \\ \hline
          & & 25 & 30 & 35 & 40 & 45 & 50 & 55 & 60 & 65 & 70 & 75 & 80 \\ \hline \hline
          & 2 & 0 & 0 & 0 & 2 & 3 & 3 & 2 & 1 & 1 & 0 & 0 & 0 \\ \cline{2-14}
          データ数(個) & 1 & 1 & 1 & 1 & 1 & 1 & 3 & 3 & 2 & 2 & 2 & 2 & 2 \\ \cline{2-14}
          & 0 & 0 & 0 & 0 & 0 & 0 & 0 & 2 & 2 & 2 & 2 & 2 & 2 \\ \cline{2-14}
          & 0 & 0 & 0 & 0 & 0 & 0 & 1 & 1 & 1 & 1 & 1 & 1 & 1 \\ \cline{2-14}
          \hline
        \end{tabular}
      }
\end{table}

\subsection{提案方式の評価}\label{subsec: eval-digest}
今までの実験により, それぞれの提案方式の適切な閾値がわかった. 次に, その適切な閾値を適用したそれぞれの提案方式を比較し, P2Pライブストリーミングにおいてダイジェストを生成する際に最も適当なダイジェスト生成方式を決定する.

3つの動画に対して, それぞれの提案方式の適合率を計算することによって評価する. 適合率は式\ref{siki:tekigo}に従う. これは, 提案方式がダイジェストとして選出したもののうち, 元々用意しておいた正解シーンが含まれている確率を意味する. 適合率の値が高いほど, より適切なダイジェスト生成方式であると言える. 表\ref{tbl:tekigo}に実験の結果を示す. 実験の結果, 最小二乗法に基づくダイジェスト生成方式の適合率が最も高く, 平均で66\%を超えるという結果になった.

\begin{eqnarray}
  適合率 = \frac{正解シーン数}{方式による抽出シーン数}
  \label{siki:tekigo}
\end{eqnarray}

\begin{table}[h]
  \caption{適合率の結果}
  \label{tbl:tekigo}
  \centering
      {\small
        \begin{tabular}{|l|l|l|l|} \hline
          & \multicolumn{3}{|c|}{適合率} \\ \hline
          生成方式 & 増減率に基づく方式 & 前後比較に基づく方式 & 最小二乗法に基づく方式 \\ \hline
          動画1 & 12.5\% & 50.0\% & 62.5\% \\ \hline
          動画2 & 37.5\% & 50.0\% & 75.0\% \\ \hline
          動画3 & 50.0\% & 50.0\% & 62.5\% \\ \hline
        \end{tabular}
      }
\end{table}

\subsection{ダイジェスト生成方式に対する評価の考察}
ダイジェスト生成方式の実験では, まずそれぞれの提案方式の適切な閾値を決定した. 次にその閾値を適用した提案方式の適合率を計算することによって最も適切なダイジェスト生成方式を決定した.

適切な閾値の決定では, 「増減率に基づくダイジェスト生成方式」, 「前後比較に基づくダイジェスト生成方式」, 「最小二乗法に基づくダイジェスト生成方式」のそれぞれで閾値を決定した. それぞれについて考察していく. 次に適合率の結果についての考察を行う.

\subsubsection{増減率に基づくダイジェスト生成方式に対する考察}
実験的に求めた望ましい閾値は増加率0.1, 減少率0.03, 一定時間3(ms)という結果であった. 増加率は減少率よりも高い値であった. これは, 瞬間的に多くのユーザがコメントをして, その瞬間に対して反応したことを意味している. 瞬間的に多くの反応があったということは, とても注目度が高いことを意味しており, ダイジェストにすべき瞬間であったことがわかる. まだ, 一定時間は3という結果であり, 長くもなく短くもない値であった. これは, 短すぎると雑音となる部分まで余計に反応してしまうためであり, 逆に長すぎると瞬間的な変化に反応出来ないからであると考えられる.

\subsubsection{前後比較に基づくダイジェスト生成方式に対する考察}
実験的に求めた望ましい閾値は増加率0.1, 減少率0.03, 一定時間4(ms)という結果であった. この閾値は「増減率に基づくダイジェスト生成方式」の結果とほぼ同じであり, 基本的には同様の考え方が出来る. しかし, こちらの方式では増加率が0.05の場合だとほとんど正解シーンを選出出来なかったことがわかる. これは「前後比較に基づくダイジェスト生成方式」の方では特に増加率の値に敏感であり, 細かい値を設定することが望ましいことがわかった.

\subsubsection{最小二乗法に基づくダイジェスト生成方式に対する考察}
実験的に求めた望ましい閾値は角度55(度), データ数3(個)という結果であった. 全体的に角度は急なほど正解シーンを多く選出していることがわかる. これは, より瞬間的な状態を検出していることがわかる. データ数は少ない方が多く正解シーンを選出していることがわかる. これは, データ数が多いほどグラフをなだらかにしてしまい, 瞬間的な状態を検出出来なかったからではないかと考えられる.

\subsubsection{適合率の結果に対する考察}
適合率は「最小二乗法に基づくダイジェスト生成方式」が最も適切であることがわかった. これは, その他の方式では雑音となる突発的な瞬間を誤検知してしまったのに対し, こちらの方式では最小二乗法を適用していることで, より自然な盛り上がり部分をダイジェストとして捉え, 検出出来ているからではないかと考えられる.

\subsubsection{ダイジェスト生成方式の要求条件に対する考察}
ダイジェスト生成方式の要求条件に対する考察を行う. 各要求条件を示し, それについての考察を行う.

\begin{itemize}
\item 特定の種類のコンテンツに依存しないダイジェスト生成方式であること \\
本研究では, 単位時間当たりにコメントしたユーザ数を利用したダイジェスト生成方式を提案し, これは特定のコンテンツに依存しない, 汎用的なものである.
\item P2Pライブストリーミングに特化したダイジェスト生成方式であること \\
既存手法では, ダイジェストを生成するための専用のサーバを利用する必要があったが, 本研究では各ノードがダイジェストを生成するためP2Pに特化したものである. また,上記と同様に参加ユーザが配信内容に対して自由にコメント投稿が出来るシステムにおいて, 単位時間当たりにコメントしたユーザ数を利用したことは, ライブストリーミングに特化したものである.
\item 生成されたダイジェストの映像によってコンテンツの内容の全体把握が出来ること \\
\ref{subsec: eval-digest}節の結果, 平均で66\%の適合率を記録した. 提案したダイジェスト生成方式によって, コンテンツの内容の全体を把握することが出来る.
\end{itemize}

\newpage

\section{トポロジ設計に対する評価}
トポロジ設計に対する評価を行う. 評価はシミュレーションで行った. シミュレーションにはNS-2を用いた. 提案システムでは各ノードに役割を与えたトポロジを設計した. その各役割の適切な割合についての評価を行う. また, 役割ノードが適切に機能しているかの評価を行う. そして, 配信者ノードから各ノードへ送られるストリームの遅延に対する評価を行う. さらに, P2Pネットワーク内に十分なダイジェストを保有できているかを評価する.

\subsection{前提条件}
シミュレーションを行う上でいくつかの前提となる条件について述べる. 前提となる項目に「帯域幅分布」, 「コメント数分布」, 「ノード数とクラスタ」がある. それぞれについての前提条件を決定する.

全ノードには固有の帯域幅を割り当てる.
「帯域幅分布」は, 他のP2Pライブストリーミング研究\cite{band-dist}で使用されている値を参考にする. この研究ではインターネットユーザの帯域幅分布を解析した2つの研究\cite{band-dist-1}\cite{band-dist-2}を参考にしている. 帯域幅は全10種類あり, 平均は540kbpsである. 表\ref{tbl:band-dist}のように決定する.

\begin{table}[h]
  \caption{帯域幅分布}
  \label{tbl:band-dist}
  \centering
      {\small
        \begin{tabular}{|c|c|c|c|c|c|c|c|c|c|c|} \hline
          \shortstack{帯域幅 \\ (kbps)} & 256 & 320 & 384 & 448 & 512 & 640 & 768 & 1024 & 1500 & 3000 \\ \hline
          \shortstack{割合 \\ (\%)} & 10.0 & 14.3 & 8.6 & 12.5 & 2.2 & 1.4 & 6.6 & 28.1 & 1.4 & 14.9 \\ \hline
        \end{tabular}
      }
\end{table}

また, 全ノードには固有のコメント数を割り当てる. サンプルとして1つの動画に対して分析を行った結果を利用する\cite{comment}. 1つ24分で2013年3月20日時点での動画を参考にしている. 1人当たりの平均コメント数は4で標準偏差は6.8である.

図\ref{fig:comment-res}は, 動画を分析した結果である. 横軸が1人当たりのコメント数, 縦軸がそのコメント数をしている人の割合である.

\begin{figure}[h]
  \centering
  \includegraphics[width=1\hsize]{fig/comment-res.eps}
  \caption{コメント数分析の結果}
  \label{fig:comment-res}
\end{figure}

\newpage

図\ref{fig:comment-res}を10種類の具体的な数字に表すと表\ref{tbl:comment-dist}が出来上がる. この表を元に, 各ノードに対して固有のコメント数を割り当てる.

\begin{table}[h]
  \caption{コメント数分布}
  \label{tbl:comment-dist}
  \centering
      {\small
        \begin{tabular}{|c|c|c|c|c|c|c|c|c|c|c|} \hline
          \shortstack{コメント数} & 2 & 5 & 7 & 10 & 12 & 15 & 17 & 20 & 22 & 25 \\ \hline
          \shortstack{割合 \\ (\%)} & 15 & 16 & 16 & 15 & 13 & 11 & 10 & 2 & 1 & 1 \\ \hline
        \end{tabular}
      }
\end{table}

また, 全ノード数に応じたクラスタの数を予め設定しておく. 設定する値は\cite{cluster-dist}の研究で示されている線形関数を利用する. 図\ref{fig:node-cluster}に参加ノード数に応じたクラスタ数の変化を示す. このグラフを元に表\ref{tbl:cluster-dist}を作成した. こちらの表の通り全ノード数に対するクラスタ数を決定する. 全ノード数は, 「ゲートノード」, 「セミゲートノード」, 「ダイジェストノード」, 「ノーマルノード」の4種類を対象とする.

\begin{figure}[h]
  \centering
  \includegraphics[width=1\hsize]{fig/node-cluster.eps}
  \caption{コメント数分析の結果}
  \label{fig:node-cluster}
\end{figure}

\begin{table}[h]
  \caption{ノード数とクラスタ数対応表}
  \label{tbl:cluster-dist}
  \centering
      {\small
        \begin{tabular}{|c|c|c|c|c|} \hline
          \shortstack{ノード数} & 200 & 400 & 600 & 800 \\ \hline
          \shortstack{クラスタ数} & 7 & 10 & 14 & 18  \\ \hline
        \end{tabular}
      }
\end{table}

その他の前提条件として, ノード間遅延を全てのノード間で100msに設定している. またパケットロス率は0\%として考慮しないこととする. また, 参加離脱を考慮した実験においては, 10秒に1回参加と離脱を行うこととする. 参加はランダムであり, 最初はダイジェスト未取得のノーマルノードとなる. ダイジェスト未取得のノーマルノードは10秒経つとダイジェスト取得済みのダイジェストノードとなる. また, シミュレーションにおいての参加離脱はノードにパケットが流れたかどうかで判断する.

\newpage

ノード数が200の時の、役割を与えた場合のトポロジ図を図\ref{fig:topology-role}に示す. 図ではゲートノードが全体の10\%, セミゲートノードが全体の10\%, ダイジェストノードが全体の20\%となるように構成されている. 具体的な構成を表\ref{tbl:topology-ex}に示す.

\begin{figure}[h]
  \centering
  \includegraphics[width=1\hsize]{fig/topology-role.eps}
  \caption{役割を与えた場合のトポロジ図}
  \label{fig:topology-role}
\end{figure}

\newpage

\begin{table}[h]
  \caption{ノード数が200時の役割を与えた時の役割構成}
  \label{tbl:topology-ex}
  \centering
      {\small
        \begin{tabular}{|c|c|} \hline
        役割 & ノード数 \\ \hline \hline
        配信者ノード & 1 \\ \hline
        ゲートノード & 21 \\ \hline
        セミゲートノード & 21 \\ \hline
        ダイジェストノード & 42 \\ \hline
        ノーマルノード & 112 \\ \hline
        \end{tabular}
      }
\end{table}

ノード数が200の時の, 役割を与えない場合のトポロジ図を図\ref{fig:topology-no-role}に示す. 図では役割が無く, 全てがノーマルノードになっている. 具体的な構成を表\ref{tbl:topology-ex-no-role}に示す.

\begin{figure}[h]
  \centering
  \includegraphics[width=1\hsize]{fig/topology-no-role.eps}
  \caption{役割を与えない場合のトポロジ図}
  \label{fig:topology-no-role}
\end{figure}

\newpage

\begin{table}[h]
  \caption{ノード数が200時の役割を与えない時の役割構成}
  \label{tbl:topology-ex-no-role}
  \centering
      {\small
        \begin{tabular}{|c|c|} \hline
        役割 & ノード数 \\ \hline \hline
        配信者ノード & 1 \\ \hline
        ノーマルノード & 199 \\ \hline
        \end{tabular}
      }
\end{table}

以上のトポロジ構成にしたがって, 役割を与えた場合と与えない場合での比較を行う.

\subsection{適切な役割の割合の決定}
提案するトポロジに存在するノードの役割のうち「ゲートノード」, 「セミゲートノード」, 「ダイジェストノード」の役割の適切な割合を決定する. まず最初に「ダイジェストノード」の割合を, 次に「ゲートノード」と「セミゲートノード」の割合を決定する.

\subsubsection{ダイジェストノードの割合}
ダイジェストノードの割合をそれぞれ, 全ノードの10\%, 20\%, 30\%で割合を変え, スループットの値を比較した. スループットはネットワーク全体で単位時間当たりに受信したパケット数と定義する.

図\ref{fig:digest-rate}に結果を示す.

\newpage

\begin{figure}[h]
  \centering
  \includegraphics[width=1\hsize]{fig/digest-rate.eps}
  \caption{ダイジェストノードの割合を変化させた時の結果}
  \label{fig:digest-rate}
\end{figure}

実験の結果, 10\%の時の平均スループットが265.71kbps, 20\%の時の平均スループットが573.12kbps, 30\%の時の平均スループットが414.63kbpsであった. 20\%の時が最もスループットの値が大きく, 適切な値であることがわかった.

\subsubsection{ゲートノードとセミゲートノードの割合}
次にゲートノードとセミゲートノードの割合を決定する. ゲートノードとセミゲートノードは常に1対1の関係となるため, 一緒に決定する. 全ノードのそれぞれ10\%と10\%, 20\%と20\%で割合を変え, スループットの値を比較した. 図\ref{fig:gate-semi-rate}に結果を示す.

\begin{figure}[h]
  \centering
  \includegraphics[width=1\hsize]{fig/gate-semi-rate.eps}
  \caption{ゲートノードとセミゲートノードの割合を変化させた時の結果}
  \label{fig:gate-semi-rate}
\end{figure}

実験の結果, 10\%と10\%の時の平均スループットが446.33kbps, 20\%と20\%の時の平均スループットが413.48kbpsであった. どちらもそれほど変わらないが, 10\%と10\%の方がスループットの値が大きかった. しかし, グラフを見るとノード数が小さい時は10\%と10\%の時の方が値が高く, ノード数が大きい時は20\%と20\%の時の方が値が高かった.

\subsection{役割の適切性に対する評価}\label{subsec:eval-role}
次に役割を与えたことの適切性に対する評価を行う. 図\ref{fig:throughput-role}に役割を与えた場合のネットワーク全体のスループットの推移の様子を示す. udp200はノード数が200の時を意味する. 図を見ると, ノード数が200の時が一番低い値を取っているが, それ以降ノード数が大きくなってもスループットの値が低下していないことがわかる.

\begin{figure}[h]
  \centering
  \includegraphics[width=1\hsize]{fig/throughput-role.eps}
  \caption{各ノード数におけるスループットの推移}
  \label{fig:throughput-role}
\end{figure}

\newpage

図\ref{fig:throughput-no-role}に役割を与えない場合のスループットの推移の様子を示す. udp200-no-roleはノード数が200の時を意味する. 図を見ると, ノード数が200の時が一番高い値を取っているが, それ以降ノード数が大きくなるに連れてスループットの値が低下していることが分かる.

\begin{figure}[h]
  \centering
  \includegraphics[width=1\hsize]{fig/throughput-no-role.eps}
  \caption{役割を与えない場合の各ノード数におけるスループットの推移}
  \label{fig:throughput-no-role}
\end{figure}

\newpage

図\ref{fig:throughput-compare}に役割を与えた場合と与えない場合の各ノード数におけるスループットの推移の平均の比較を示す. averageが役割を与えた場合であり, average-no-roleが役割を与えない場合である. 図を見ると, 役割が無い場合はノード数が大きくなるとスループットの値が低下しているのに対し, 役割がある場合はノード数が大きくなってもスループットの値が低下していないことがわかる. また, 役割を与えない場合の平均のスループットの値は161.48kbpsであり, 役割を与えた場合の平均のスループットの値は406.73kbpsであった.

例えば, 画面サイズ678×432, フレームレート19fpsでH.264エンコードでストリーミング配信を行った場合, 目安となるビットレートは390kbpsとなる. 役割を与えた場合の平均のスループットの値は406.73kbpsなので, この値以上の性能が出ることがわかる. この結果, 役割を与えたことによってP2Pネットワーク全体で質の高い映像を見ることが可能で, 役割を与えたことの有用性を確認することが出来た.

\newpage

\begin{figure}[h]
  \centering
  \includegraphics[width=1\hsize]{fig/throughput-compare.eps}
  \caption{役割を与えない場合の各ノード数におけるスループットの推移の平均の比較}
  \label{fig:throughput-compare}
\end{figure}

\subsection{遅延に対する評価}\label{subsec:eval-delay}
遅延に対する評価を行う. 遅延に対する評価として2つの要因を考えた. 1つ目は配信者ノードから各ノードへのホップ数であり, 2つ目は各ノードの接続数である. まずはホップ数に対しての評価を行い, 次に接続数に対しての評価を行う.

表\ref{tbl:hop-count}は役割がある場合の各ノード数におけるネットワーク全体の平均ホップ数であり, 表\ref{tbl:hop-count-no-roll}は役割が無い場合の各ノード数におけるネットワーク全体の平均ホップ数である. 役割がある場合と無い場合の平均ホップ数をグラフ化したものを図\ref{fig:average-hop-count}に示す.

\begin{table}[h]
  \caption{役割がある場合の各ノード数におけるネットワーク全体の平均ホップ数}
  \label{tbl:hop-count}
  \centering
      {\small
        \begin{tabular}{|c|c|c|} \hline
        ノード数 & 合計ホップ数 & 平均ホップ数 \\ \hline \hline
        200 & 645 & 3.23 \\ \hline
        400 & 1288 & 3.22 \\ \hline
        600 & 1864 & 3.11 \\ \hline
        800 & 2514 & 3.14 \\ \hline
        \end{tabular}
      }
\end{table}

\begin{table}[h]
  \caption{役割が無い場合の各ノード数におけるネットワーク全体の平均ホップ数}
  \label{tbl:hop-count-no-roll}
  \centering
      {\small
        \begin{tabular}{|c|c|c|} \hline
        ノード数 & 合計ホップ数 & 平均ホップ数 \\ \hline \hline
        200 & 488 & 2.44 \\ \hline
        400 & 980 & 2.45 \\ \hline
        600 & 1464 & 2.44 \\ \hline
        800 & 2040 & 2.45 \\ \hline
        \end{tabular}
      }
\end{table}

\begin{figure}[h]
  \centering
  \includegraphics[width=1\hsize]{fig/average-hop-count.eps}
  \caption{各ノード数におけるネットワーク全体の平均ホップ数グラフ}
  \label{fig:average-hop-count}
\end{figure}

実験の結果, ノード数を変化させた時の平均のホップ数は役割がある場合と無い場合でそれぞれ3.18と2.45である. この結果, 役割が無い場合が役割がある場合に比べてホップ数が少なく, 優れていることがわかった.

表\ref{tbl:connect-num}は役割がある場合の各接続数におけるノード数を示しており, 表\ref{tbl:connect-num-no-roll}は役割が無い場合の各接続数におけるノード数を示している.

\newpage

\begin{table}[h]
  \caption{役割がある場合の各接続数におけるノード数}
  \label{tbl:connect-num}
  \centering
      {\small
        \begin{tabular}{|c|c||c|c|c|c|c|c|c|c|c|c|} \hline
                &     & \multicolumn{10}{|c|}{接続数} \\ \hline
                &     & 3  & 4  & 6   & 7 & 14 & 15 & 16 & 23 & 24 & 25 \\ \hline \cline{2-12}
                & 200 & 0  & 25 & 21  & 0 & 12 & 12 & 2  & 0  & 0  & 0  \\ \cline{2-12}
        ノード数 & 400 & 0  & 40 & 120 & 0 & 1  & 0  & 0  & 0  & 0  & 0  \\ \cline{2-12}
                & 600 & 14 & 56 & 108 & 0 & 9  & 1  & 0  & 10 & 0  & 0  \\ \cline{2-12}
                & 800 & 18 & 72 & 252 & 8 & 273& 76 & 11 & 47 & 38 & 5  \\ \cline{2-12}
        \hline
        \end{tabular}
      }
\end{table}

\begin{table}[h]
  \caption{役割が無い場合の各接続数におけるノード数}
  \label{tbl:connect-num-no-roll}
  \centering
      {\small
        \begin{tabular}{|c|c||c|c|c|} \hline
                &     & \multicolumn{3}{|c|}{接続数} \\ \hline
                &     & 11  & 22 & 23   \\ \hline \cline{2-5}
                & 200 & 0   & 0  & 0    \\ \cline{2-5}
        ノード数 & 400 & 400 & 10 & 0 \\ \cline{2-5}
                & 600 & 12 & 442 & 146 \\ \cline{2-5}
                & 800 & 8 & 686 & 106 \\ \cline{2-5}
        \hline
        \end{tabular}
      }
\end{table}

\newpage

また, それぞれの表をグラフにしたものを図\ref{fig:connect-num}と図\ref{fig:connect-num-no-roll}に示す.

\begin{figure}[h]
  \centering
  \includegraphics[width=1\hsize]{fig/connect-num.eps}
  \caption{役割がある場合の各接続数におけるノード数グラフ}
  \label{fig:connect-num}
\end{figure}

\newpage

\begin{figure}[h]
  \centering
  \includegraphics[width=1\hsize]{fig/connect-num-no-roll.eps}
  \caption{役割が無い場合の各接続数におけるノード数グラフ}
  \label{fig:connect-num-no-roll}
\end{figure}

また, 役割がある場合と無い場合の各接続数におけるノード数の平均の比較を図\ref{fig:connect-num-average-compare}に示す.

\newpage

\begin{figure}[h]
  \centering
  \includegraphics[width=1\hsize]{fig/connect-num-average-compare.eps}
  \caption{役割が無い場合の各接続数におけるノード数グラフ}
  \label{fig:connect-num-average-compare}
\end{figure}

実験の結果, 役割がある場合は接続数が6から14の時に集中しており, 役割が無い場合は22から23に集中していることがわかった. また, 各ノード数における平均の接続数では, 役割がある場合が役割が無い場合に比べて全体的に接続数が少なく, 優れていることがわかった.

\subsection{ダイジェスト保有率に対する評価}\label{subsec:eval-digest-have}
ダイジェスト保有率に対する評価を行う. ダイジェスト保有率に対する評価では, ノードが参加と離脱を行う場合を想定してシミュレーションを行う. このシミュレーションを行うにあたって与えた条件を次に示す. まず, これまではノード数に対するクラスタ数を表\ref{tbl:cluster-dist}のように示してきたが, このシミュレーションでは1クラスタのみを対象として行う. 具体的にはノード数28に対してクラスタ数1で行う. また, NS-2ではノードを新たに追加したり, 既存のノードを離脱させることが厳密には出来ない. そこで, このシミュレーションではノードに対してパケットを流すことでネットワークに参加していることを表し, パケットを流さないことでネットワークから離脱していることを表す. 表\ref{tbl:digest-have}にダイジェスト保有率に対する評価の構成を示す.

\begin{table}[h]
  \caption{ダイジェスト保有率に対する評価の構成}
  \label{tbl:digest-have}
  \centering
      {\small
        \begin{tabular}{|c||c|} \hline
        ノード数 & 28 \\ \hline
        クラスタ数 & 1 \\ \hline
        & 配信者ノード: 1 \\ \cline{2-2}
        & ゲートノード: 3 \\ \cline{2-2}
        役割とノード数 & セミゲートノード: 3 \\ \cline{2-2}
        & ダイジェストノード: 6 \\ \cline{2-2}
        & ノーマルノード: 16 \\ \cline{2-2}
        \hline
        \end{tabular}
      }
\end{table}

1つのダイジェストは1つのダイジェストノードが保有しており, ダイジェストノードが離脱することでそのノードが保有していたダイジェストが失われることになる. ダイジェスト保有率は以下の式\ref{siki:digest-have}に従う.

\begin{eqnarray}
        ダイジェスト保有率 = \nonumber \\
        \frac{ダイジェストノード数 - 離脱したダイジェストノード数}{ダイジェストノード数}
        \label{siki:digest-have}
\end{eqnarray}

また, 各ノードの離脱率はコメント数に比例し, 表\ref{tbl:comment-dist}に従う.

シミュレーションを10回行った結果を表\ref{tbl:digest-have-res}に示す.

\begin{table}[h]
  \caption{ダイジェスト保有率に対する評価の結果}
  \label{tbl:digest-have-res}
  \centering
      {\small
        \begin{tabular}{|c||c|c|c|c|c|c|c|c|c|c|} \hline
        回数 & 1 & 2 & 3 & 4 & 5 & 6 & 7 & 8 & 9 & 10 \\ \hline
        ダイジェスト保有率(\%) & 66.7 & 83.3 & 83.3 & 66.7 & 66.7 & 83.3 & 88.3 & 88.3 & 66.7 & 83.3 \\ \hline
        \end{tabular}
      }
\end{table}

実験の結果, 全ての回数で66.7\%以上の保有率があり, 平均で76\%の保有率を達成した. 配信内容全体の76\%を把握出来るということであり, 配信内容の概要を知るのに十分であると考えられる.

\subsection{トポロジ設計に対する評価の考察}
トポロジ設計に対する実験では, まず「ダイジェストノード」, 「ゲートノード」, 「セミゲートノード」の3つの役割のそれぞれで適切な割合を決定した. 次に役割を与えたことの適切性についての評価を行った. そして配信者ノードからの遅延に対する評価を行った. さらにダイジェスト保有率に対する評価を行った. それぞれについての考察を行う.

\subsubsection{ダイジェストノードの割合の決定に対する考察}
ダイジェストノードの割合は20\%が良いという結果だった. 図\ref{fig:digest-rate}を見ると, 20\%の時はノード数が600の時にとても高い値を記録しているが, それ以外は30\%の時とほぼ同じ値を記録していることがわかる. 20\%の時は突発的に大きな値を取っていた可能性が高く, 実際は状況に応じて20\%から30\%の割合にするのが良いと考えた.

\subsubsection{ゲートノードとセミゲートノードの割合の決定に対する考察}
ゲートノードとセミゲートノードの割合はノード数によって割合を変えることが良いという結果であった. ノード数が小さい時は10\%と10\%に, ノード数が大きい時は20\%と20\%が良いということだった. ノード数が小さい時にそれぞれの割合が小さい方が良い理由として, 1ノード当たりの接続数が少ないことが原因であると考えられる. ノード数が小さいと接続数が少ないので, 1ノード当たりの負担が小さく, 小さい割合でも問題ないのではないかと考えられる. ノード数が大きい時にそれぞれの割合が大きい方が良い理由も同様である. 1ノード当たりの負担が大きくなり, 割合を増やす必要があるのではないかと考えられる.

\subsubsection{役割の割合決定全体に対する考察}
「ダイジェストノード」と, 「ゲートノード」と「セミゲートノード」の割合のどちらも状況によって変える方が良いという結果だった. とくに後者では具体的に接続数が問題になっていた. 本システムでは役割を割合で決定しているが, 接続数に関しては考慮していなかった. ノード数によってクラスタの数を変えているので, 接続数が爆発的に多くなることはないが, 役割の数を決定する方法として割合よりも接続数を考慮すべきであったと考えられる.

\subsubsection{役割を与えた場合に対する考察}
役割を与えた場合は, ノード数が200の時以外は高い値を取っているという結果だった. ノード数が200の時に低くなっている理由として, 1ノード当たりの接続数が少なすぎるからであると考えられる. 十分な接続数を確保できないため, 各ノードが取得可能なパケットが少なく, 全体としてスループットの値が低くなっているのではないかと考えられる.

\subsubsection{役割を与えない場合に対する考察}
役割を与えない場合は, ノード数が200の時にスループットが一番高い値を取っているという結果だった. ノード数が多くなるに連れてスループットの値が低くなっていく理由として, 接続数が爆発的に多くなるからではないかと考えられる. 役割を与えない場合は全てのノードがノーマルノードであるため, 接続数が分散されずに多くなってしまう. そのためスループットの値が小さくなっていると考えられる.

\subsubsection{役割の適切性全体に対する考察}
役割を与えたほうが役割を与えない場合に比べ全体的なスループットの値が大きく, 役割を与えることは有用であったという結果だった. 値に差が出てしまった要因としてはやはり接続数の問題があった. 特に役割を与えない場合では接続数の多さがスループット低下の要因であることを顕著に表していた. 結果的に役割を与えることは有用であることはわかったが, さらに性能をあげるためには接続数を考慮すべきであることがわかった.

\subsubsection{遅延に対する考察}
ホップ数と接続数の2つについて実験を行った. 役割を与えた場合は与えない場合に比べて, ホップ数が多く劣っているが, 接続数は少なく優れているという結果だった. 上記で示した考察により, 役割を与えた場合は与えない場合に比べてスループットの値が大きく, 有用であった. その結果, 接続数の数がスループットにも影響していると考えられ, 同時に遅延にも影響していると考えられる.

\subsubsection{ダイジェスト保有率に対する考察}
ダイジェスト保有率は平均で76\%を記録し, 概要を知るのに十分という結果だった. ダイジェスト保有率が高かった理由として, ダイジェストノードをコメント数の多いノードから選出したことがあげられる. コメント数の多いノードは離脱可能性が低く, そのようなノードにダイジェストを持たせることで, ダイジェストをP2Pネットワーク内に保持出来ると考えられる. しかし, 本研究では1ダイジェストノードに1ダイジェストを持たせており, ダイジェストノードが抜けると, そのノードが保持していたダイジェストが失われてしまう. この問題を解決するために, 1つのダイジェストは複数のダイジェストで保持することが良いと考えられる. このようにすることで, 接続数は多くなるものの, ダイジェスト保有率はより高く維持することが出来るはずである.

\subsubsection{トポロジ設計の要求条件に対する考察}
トポロジ設計の要求条件に対する考察を行う. 各要求条件を示し, それについての考察を行う.

\begin{itemize}
\item ライブストリーミングの映像を見られるだけの性能があること \\
\ref{subsec:eval-role}節の結果, 役割を与えた場合では, ノード数を変化させた時の平均のスループットの値が406.73kbpsであった. これはライブストリーミングの映像を十分に見ることが出来る性能である.
\item 少ない遅延でパケットが届くこと \\
\ref{subsec:eval-delay}節の結果, 接続数が遅延に影響していることがわかった. 役割を与えた場合では, ノード数が大きくなっても接続数が低く抑えられ, 少ない遅延でパケットが各ノードに届くと考えられる.
\item ダイジェストをP2Pネットワーク内に確保できること \\
\ref{subsec:eval-digest-have}節の結果, ダイジェスト保有率は平均で76\%を達成した これは配信内容の概要を知るのに十分なダイジェストをP2Pネットワーク内に確保出来ると考えられる.
\end{itemize}



\clearpage
\chapter{まとめと今後の課題} \label{txt:conclusion}
P2Pライブストリーミング提案、ダイジェスト生成方式とトポロジを提案、シミュレーションの結果よかった

今後、他のP2Pライブストリーミングとの比較、実際にソフトウェアとして動作、有用性を検証



\clearpage

\ackn{
本研究を遂行するにあたり、様々な方々にお世話になりました。

指導教員である末田欣子先生には, 仕事が忙しいにも関わらず毎週行われたゼミでの終始熱心なご指導を頂きました. また, ゼミ以外でもメール等を介して常に適切な助言を賜りました. ここに厚く御礼申し上げます.

講座内進捗や講座内輪講などにおいて研究を進める上での貴重な意見や助言を頂いた, 多田好克先生, 小宮常康先生, 鶴岡行雄先生, 本庄利守先生には深く感謝します.

研究室の先輩である中原祥吾様, 他の研究室の先輩である藤田竜一様, 村井栄王様, 森中翔太郎様, 片桐国建様, 石田峰文様, 神保直幸様, 若井英之様, 講座の同期である熊谷佑弥様, 山本峻丸様, 吉原大夢様, 磯谷俊明様, 工藤朋哉様, 講座の後輩であるグエン クアン ヒエップ様, 木田純平様, 関湧大様, ゾリーグ ウンダラム様, 李明元様, 新タ智啓様とは日々の研究室生活を共に過ごさせて頂き, 多くの助言を頂きました. ここに感謝の意を表します.

さらに, 私が電気通信大学大学院で様々なことを学ぶことが出来たことは, 両親, 友人, 先生, 先輩, 後輩など, これまでの人生で出会った方々のおかげです. 皆様への心から感謝の気持ちと御礼を申し上げたく、謝辞にかえさせていただきます。

\begin{flushright}
  2015年1月吉日 鈴木駿介
\end{flushright}
}

\clearpage

%
%       ・著者が2人の場合は A and B、複数の場合は A, B, C, and D と書く
%       ・著者名と論文タイトルは : で区切る
%       ・論文タイトルは`` '' で囲む
%       ・論文誌名は斜字体で書く
%       ・図書を参照する場合は ISBN 番号も書く
%       ・URL はなるべく参考文献に入れない(資料がそこでしか手に入らない
%         場合はやむを得ないが)
%
\begin{thebibliography}{99}
%\addcontentsline{doc}{section}{参考文献}

\bibitem{Akito05}
門田暁人 and Clark Thomborson:
``test''
{\it IPSJ Magazine}, vol. 46, no. 4, pp. 431--437, 2005.

\bibitem{nico}
  ニコニコ生放送, http://live.nicovideo.jp/, 2015年閲覧.
\bibitem{twi}
  Twicasting, http://twitcasting.tv/, 2015年閲覧.
\bibitem{ust}
  Ustream, http://www.ustream.tv/, 2015年閲覧.
\bibitem{afr}
  AfreecaTV, http://www.afreeca.com/, 2015年閲覧.
\bibitem{hcps}
  Yang Guo, Chao Liang, Yong Liu, “Hierarchically Clustered P2P Video Streaming: Design, implementation, and evaluation”, Computer Networks, pp.3432-3445, 2012.
\bibitem{chojo}
  元橋 智紀, 藤本 章宏, 廣田 悠介, 戸出 英樹, 村上 孝三, “多様な配信木により離脱耐性と遅延抑制を向上させる重畳クラスタ木型動画配信システム”, 通信技術の革新を担う学生論文特集, p.132-142, 2014年.
\bibitem{metree}
  Huey-Ing Liu, その他, “MeTree: A Contribution and Locality-Aware P2P Live Streaming Architecture”, AINA 24th IEEE International Conference on, pp.1136-1143, 2010.
\bibitem{dis}
  V. Padmanabhan, H. Wang, P. Chou, and K. Sripanidkulchai, “Distributing streaming me-
dia content using cooperative networking,” Proc.NOSDAV’02, pp.177–186, May 2002.
\bibitem{streamline}
  G. Bianchi, N. Melazzi, L. Bracciale, F. Piccolo, and S. Salsano, “Streamline: An optimal distribution al-gorithm for peer-to-peer real-time streaming,” IEEE Trans. Parallel Distrib. Syst., vol.21, no.6, pp.857–871, June 2010.
\bibitem{pdms}
  橋本 隆子, 加登 岡隆, 飯沢 篤志, ”スポーツ映像におけるシーン重要度算出アルゴリズムとその評価”, 信学会DEWS2003.
\bibitem{yakyu}
  熊野 雅仁, 有木 康雄, 塚田 清志, “野球中継のハイライトシーン実時間配信を目的とした特徴のマイニングによるPCシーンの自動検出”, 映像情報メディア学会誌, vol.59, No.1, pp.77-84, 2005.
\bibitem{band-dist}
  Zhengye Liu, Yanming Shen, Ross K.W., Panwar S.S., Yao Wang, “Substream Trading: Towards an open P2P live streaming system”, Network Protocols, ICNP, pp.94-130, 2008.
\bibitem{cluster-dist}
  大村淳己,高田和也,後藤滋樹, “Location Based Clusteringを用いたP2Pストリーミング,電子情報通信学会技術研究報告”, vol. 110, no. 373, IN2010-124, pp.37-42, 2011.
\bibitem{comment}
  “ニコニコ動画のコメントを分析してみる@2013年冬アニメ”, http://blog.livedoor.jp/mgpn/archives/51887779.html, 2014年閲覧.


\end{thebibliography}


\appendix
%% \chapter{シミュレーションで使用したコード}

\section{役割有無実験}

\subsection{役割有りメインコード}
\label{txt:not-join-leave-my-goddard}
\lstinputlisting[
  language=tcl,
  escapechar={},
  basicstyle=\tiny,
  numbers=left,
  breaklines=true,
  caption=役割有りメインコード,
  label=code:not-join-leave-my-goddard
]
{code/not-join-leave/my-goddard.tcl}

\subsection{役割無しメインコード}
\label{txt:not-join-leave-my-goddard-no-roll}
\lstinputlisting[
  language=tcl,
  escapechar={},
  basicstyle=\tiny,
  numbers=left,
  breaklines=true,
  caption=役割無しメインコード,
  label=code:not-join-leave-my-goddard-no-roll
]
{code/not-join-leave/my-goddard-no-roll.tcl}

\subsection{外部プロシージャ共通コード}
\label{txt:not-join-leavemy-goddard-procs}
\lstinputlisting[
  language=tcl,
  escapechar={},
  basicstyle=\tiny,
  numbers=left,
  breaklines=true,
  caption=外部プロシージャ共通コード,
  label=code:not-join-leave-my-goddard-procs
]
{code/not-join-leave/my-goddard-procs.tcl}

\subsection{デフォルトパラメータコード}
\label{txt:not-join-leave-my-goddard-default}
\lstinputlisting[
  language=tcl,
  escapechar={},
  basicstyle=\tiny,
  numbers=left,
  breaklines=true,
  caption=デフォルトパラメータコード,
  label=code:not-join-leave-my-goddard-default
]
{code/not-join-leave/my-goddard-default.tcl}

\section{参加離脱実験}

\subsection{メインコード}
\label{txt:join-leave-low-data-my-goddard}
\lstinputlisting[
  language=tcl,
  escapechar={},
  basicstyle=\tiny,
  numbers=left,
  breaklines=true,
  caption=メインコード,
  label=code:join-leave-low-data-my-goddard
]
{code/join-leave-low-data/my-goddard.tcl}

\subsection{外部プロシージャコード}
\label{txt:join-leave-low-data-my-goddard-procs}
\lstinputlisting[
  language=tcl,
  escapechar={},
  basicstyle=\tiny,
  numbers=left,
  breaklines=true,
  caption=外部プロシージャコード,
  label=code:join-leave-low-data-my-goddard-procs
]
{code/join-leave-low-data/my-goddard-procs.tcl}

\subsection{デフォルトパラメータコード}
\label{txt:join-leave-low-data-my-goddard-default}
\lstinputlisting[
  language=tcl,
  escapechar={},
  basicstyle=\tiny,
  numbers=left,
  breaklines=true,
  caption=デフォルトパラメータコード,
  label=code:join-leave-low-data-my-goddard-default
]
{code/join-leave-low-data/my-goddard-default.tcl}



\end{document}
