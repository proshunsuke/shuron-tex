ネットワーク技術の発展に伴い, 近年ではリアルタイム動画配信サービスが人気となっている. このようなサービスはサーバから各クライアントに配信する, サーバ-クライアント方式が一般的であるが, 配信サーバのコスト削減などの理由で, サーバを介さずに直接クライアント同士で配信を行うP2Pを利用したライブストリーミング配信が期待されている. しかし, 既存のP2Pライブストリーミングシステムでは, 途中参加したユーザはそれまでの配信内容を把握することが出来ないといった問題がある. そこで本研究ではP2Pライブストリーミングにおいて, 途中参加したユーザがダイジェスト視聴可能なP2Pライブストリーミングシステムを提案する. 本研究では2つの段階を考えた. 1つ目はP2Pネットワーク内でダイジェストを生成することである. 2つ目は作成したダイジェストをP2Pネットワーク内で保持し, 広めるためのトポロジ設計を行うことである.

1つ目の段階ではダイジェスト生成方式を提案した. 具体的には「閾値に基づくダイジェスト生成方式」, 「前後比較に基づくダイジェスト生成方式」, 「最小二乗法に基づくダイジェスト生成方式」の3つを提案した. システムは参加ユーザが自由にコメント投稿出来ることを想定しており, 3つの方式ともコメントをした参加ユーザ数を利用した. 「閾値に基づくダイジェスト生成方式」ではユーザ数が増加している時を始点, 減少している時を終点とし, 始点から終点で最もユーザ数の多い点をダイジェストとした. 「前後比較に基づくダイジェスト生成方式」ではある点についてユーザ数が前後で急激に増減している点をダイジェストとした. 「最小二乗法に基づくダイジェスト生成方式」では最小二乗法を適用して2直線の重なりが鋭角な点をダイジェストとした. 実験では, ある動画に対する適合率で評価を行った. その結果「最小二乗法に基づくダイジェスト生成方式」が最も適合率が高く, 適切な方式であることがわかった.

2つ目の段階ではトポロジを提案した. 具体的には複数クラスタ型において, 各ノードに役割をもたせたトポロジを設計した. 役割は「ゲートノード」, 「セミゲートノード」, 「ダイジェストノード」が存在する. ダイジェスト生成方式ではコメントをしたユーザ数を利用したが, トポロジ設計では各ノードの役割を決定するのに各コメント数におけるユーザ数の割合を利用した. ゲートノードは配信者ノードから配信されたコンテンツのパケットをクラスタ内で一番最初に取得する役目を担う. 高帯域かつコメント数の多いノードが選出される. セミゲートノードはクラスタ内でゲートノードから受信したパケットをクラスタ内部に拡散する役目を担う. ゲートノードの次に高帯域でコメント数の多いノードが選出される. ダイジェストノードはダイジェストの作成, また作成したダイジェストを保有し, 新規参加ノードへ送信する役目を担う. 最もコメント数の多いノードが選出される. 実験ではNS-2というシミュレーションを使って評価を行った. 役割の適切性に対する評価では, 役割を与えない場合ではノード数が増えるに従ってスループットの値が下がっていくが, 役割を与えた場合ではスループットの値が下がること無く, ノード数が増えても継続的に高い能力を示しており, 役割を与えることの有用性を確認することが出来た.
