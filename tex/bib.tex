%
%       ・著者が2人の場合は A and B、複数の場合は A, B, C, and D と書く
%       ・著者名と論文タイトルは : で区切る
%       ・論文タイトルは`` '' で囲む
%       ・論文誌名は斜字体で書く
%       ・図書を参照する場合は ISBN 番号も書く
%       ・URL はなるべく参考文献に入れない(資料がそこでしか手に入らない
%         場合はやむを得ないが)
%
\begin{thebibliography}{99}
%\addcontentsline{doc}{section}{参考文献}

\bibitem{Akito05}
門田暁人 and Clark Thomborson:
``test''
{\it IPSJ Magazine}, vol. 46, no. 4, pp. 431--437, 2005.

\bibitem{nico}
  ニコニコ生放送, http://live.nicovideo.jp/, 2015年閲覧.
\bibitem{twi}
  Twicasting, http://twitcasting.tv/, 2015年閲覧.
\bibitem{ust}
  Ustream, http://www.ustream.tv/, 2015年閲覧.
\bibitem{afr}
  AfreecaTV, http://www.afreeca.com/, 2015年閲覧.
\bibitem{hcps}
  Yang Guo, Chao Liang, Yong Liu, “Hierarchically Clustered P2P Video Streaming: Design, implementation, and evaluation”, Computer Networks, pp.3432-3445, 2012.
\bibitem{chojo}
  元橋 智紀, 藤本 章宏, 廣田 悠介, 戸出 英樹, 村上 孝三, “多様な配信木により離脱耐性と遅延抑制を向上させる重畳クラスタ木型動画配信システム”, 通信技術の革新を担う学生論文特集, p.132-142, 2014年.
\bibitem{metree}
  Huey-Ing Liu, その他, “MeTree: A Contribution and Locality-Aware P2P Live Streaming Architecture”, AINA 24th IEEE International Conference on, pp.1136-1143, 2010.
\bibitem{dis}
  V. Padmanabhan, H. Wang, P. Chou, and K. Sripanidkulchai, “Distributing streaming me-
dia content using cooperative networking,” Proc.NOSDAV’02, pp.177–186, May 2002.
\bibitem{streamline}
  G. Bianchi, N. Melazzi, L. Bracciale, F. Piccolo, and S. Salsano, “Streamline: An optimal distribution al-gorithm for peer-to-peer real-time streaming,” IEEE Trans. Parallel Distrib. Syst., vol.21, no.6, pp.857–871, June 2010.
\bibitem{pdms}
  橋本 隆子, 加登 岡隆, 飯沢 篤志, ”スポーツ映像におけるシーン重要度算出アルゴリズムとその評価”, 信学会DEWS2003.
\bibitem{yakyu}
  熊野 雅仁, 有木 康雄, 塚田 清志, “野球中継のハイライトシーン実時間配信を目的とした特徴のマイニングによるPCシーンの自動検出”, 映像情報メディア学会誌, vol.59, No.1, pp.77-84, 2005.
\bibitem{comment}
  “ニコニコ動画のコメントを分析してみる@2013年冬アニメ”, http://blog.livedoor.jp/mgpn/archives/51887779.html, 2014年閲覧.

\end{thebibliography}
