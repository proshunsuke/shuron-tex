\section{結論}
本研究では, P2Pライブストリーミングにおいて, 途中参加したユーザがダイジェスト視聴可能なP2Pライブストリーミングシステムを提案した. システムを実現するために2つの段階を考えた. 1つ目はP2Pネットワーク内でダイジェストを生成することで, 2つ目は作成したダイジェストをP2Pネットワーク内で保持し, 広めるためのトポロジ設計を行うことである.

1つ目の段階ではダイジェスト生成方式を提案した. 3つの方式を提案し, それぞれのダイジェストの適合率を比較した結果, 「最小二乗法に基づくダイジェスト生成方式」が最も適切で, 映像全体の66\%以上のダイジェストを判定出来ることが分かった.

2つ目の段階ではトポロジを提案した. 複数クラスタ型において, 各ノードに役割を持たせたトポロジを設計した. シミュレーションでは, 役割を持たせた場合と役割を持たせない場合のネットワーク全体のスループットの値を比較した. その結果, 役割を持たせた場合の平均のスループットの値は406.73kbpsと高く, またノード数が大きくなってもスループットの値は下がらなかった. 役割を持たせることの有用性を確認することが出来た.

\section{今後の課題}
ダイジェスト生成方式においては, どのような種類にも対応可能な汎用的なダイジェスト生成方式を目指していた. しかし, 本研究では3種類の動画を対象とした実験しか出来なかった. 今後はより多くの種類の動画に対して方式を適用して有用性の確認をしていく予定である.

また, トポロジ設計では既存のP2Pライブストリーミングシステムとは定性的な評価しか行えなかった. 同じ条件下でシミュレーションを行い, 他研究との比較を行うべきである. さらに, 本研究ではシミュレーション上での評価にとどまっている. 実際のP2Pネットワーク上でシステムを動かし, 有用性の検証をすべきである.

