本研究では,  配信に途中参加したユーザがダイジェスト視聴可能なP2Pライブストリーミングシステムを提案してきた. そのために2つの段階を考え, 1つ目の段階では, P2Pネットワーク内で特別なサーバに頼らない, 汎用的なダイジェスト生成方式を提案した. 2つ目の段階では, 1つ目の段階において生成されたダイジェストをP2Pネットワーク内で保持し, それを拡散させるためのトポロジ設計を提案した.

この章では, まず1段階目のダイジェスト生成方式に対する評価を行い, その次に2段階目のトポロジ設計に対する評価を行う.

ダイジェスト生成方式では, まず3つの提案方式それぞれの適切な閾値を決めるための評価を行う. その次に実際の動画コンテンツを対象に提案システムを適応し, ダイジェスト生成に最も適切な方式を決定する. トポロジ設計では, まず各役割の適切な割合を決めるための評価を行う. その次にその役割が適切な役目を果たしているかの評価を行う. さらに生成されたダイジェストが適切にP2Pネットワーク内で保持され, 視聴出来たかの評価を行う.

表\ref{tbl:env}に評価実験を行った環境を示す.

\begin{table}[h]
  \caption{実行環境}
  \label{tbl:env}
  \centering
      {\small
        \begin{tabular}{|l|l|l|} \hline
          環境名 & 規格 & バージョン \\ \hline \hline
          OS & Ubuntu & 12.04 64-bit  \\ \hline
          CPU & Intel Core i7 2.10~GHz & \\ \hline
          メモリ & 8~GB & \\ \hline
          シミュレータ & NS2 & 2.35 \\ \hline
          言語 & OTcl & 1.14 \\ \cline{2-3}
           & TK & 8.5.10 \\ \hline
          可視化ツール & nam & 1.15 \\ \hline
        \end{tabular}
      }
\end{table}

\section{ダイジェスト生成方式に対する評価}
\subsection{評価の準備}
ダイジェスト生成方式に対する評価を行う際の準備について述べる. 評価の1つ目は, 3つの提案方式の適切な閾値を決めることである. 2つ目は3つの提案方式のうち最も適切な方式を決定することである.

2つの評価に共通するものとして, 対象となる映像がある. まず時間の長さの異なる3つ映像に対して, 時間の10分の1となる1つ2分間の正解シーンを用意した. つまり, 例えば100分の映像を扱う際には5個の正解シーンが存在するということである.

\subsection{適切な閾値の決定}
3つの提案方式の適切な閾値を決定するための評価方法について述べる. まず, 提案システムがダイジェストであると選出したシーンの数が, 元々用意しておいた正解シーンよりも半分以上であるものを集計する. なお, 正解シーンの前後2分間をダイジェストと判断した場合を選出出来たと判断して集計する. 集計した後, 最も多く正解シーンを選出した閾値を適切な値とする. 例えば100分の映像ならば, 正解シーンが5つ存在する. そのうち提案システムが半分以上選出出来たらならばプラス1カウントする. これを3つの映像それぞれについて行い, 最も多く選出出来た(映像は3種類なので最大で3)閾値を適切な値とする.

\subsubsection{閾値に基づくダイジェスト生成方式の閾値}
「閾値に基づくダイジェスト生成方式」では式\ref{fig:sikiiti}において閾値は, 増加率($Th_{inc}$), 減少率($Th_{dec}$), 一定期間($T$)であった. 図\ref{fig:digest1-1}は閾値に基づくダイジェスト生成方式の抽出例である. 時刻に付いている赤い丸が正解シーンであり, 縦に緑の線がシステムがダイジェストであると判断して選出した箇所である. 図を見ると2箇所正解していることがわかる.

表\ref{tbl:digest1-1}に増加率0.05の時の正解シーン選出の結果を, 表\ref{tbl:digest1-2}に増加率0.10の時の正解シーン選出の結果を示す. 実験の結果, 増加率が0.1, 減少率が0.03, 一定期間が3である時が一番正解数が多く, 適切な値であることがわかった.

\begin{figure}[h]
  \centering
  \includegraphics[width=1\hsize]{fig/digest1-1.eps}
  \caption{閾値に基づくダイジェスト生成方式の抽出例}
  \label{fig:digest1-1}
\end{figure}

\begin{table}[h]
  \caption{増加率0.05の時の正解シーン選出の結果}
  \label{tbl:digest1-1}
  \centering
      {\small
        \begin{tabular}{|l|l||l|l|l|} \hline
           \multicolumn{5}{|c|}{増加率($Th_{inc}$)0.05} \\ \hline
           & & \multicolumn{3}{c|}{減少率($Th_{dec}$)} \\ \hline
           & & 0.03 & 0.05 & 0.10 \\ \hline \hline \cline{2-5}
           & 1 & 0 & 1 & 1 \\ \cline{2-5}
           & 2 & 0 & 0 & 2 \\ \cline{2-5}
           & 3 & 2 & 2 & 1 \\ \cline{2-5}
           一定期間($T$)& 4 & 1 & 1 & 1 \\ \cline{2-5}
           & 5 & 2 & 2 & 1 \\ \cline{2-5}
           & 6 & 2 & 2 & 1 \\ \cline{2-5}
           & 7 & 1 & 1 & 1 \\ \cline{2-5}
           \hline
        \end{tabular}
      }
\end{table}

\newpage

\begin{table}[h]
  \caption{増加率0.10の時の正解シーン選出の結果}
  \label{tbl:digest1-2}
  \centering
      {\small
        \begin{tabular}{|l|l||l|l|l|} \hline
           \multicolumn{5}{|c|}{増加率($Th_{inc}$)0.10} \\ \hline
           & & \multicolumn{3}{c|}{減少率($Th_{dec}$)} \\ \hline
           & & 0.03 & 0.05 & 0.10 \\ \hline \hline \cline{2-5}
           & 1 & 1 & 1 & 0 \\ \cline{2-5}
           & 2 & 2 & 2 & 1 \\ \cline{2-5}
           & 3 & 3 & 2 & 1 \\ \cline{2-5}
           一定期間($T$)& 4 & 2 & 2 & 1 \\ \cline{2-5}
           & 5 & 2 & 2 & 1 \\ \cline{2-5}
           & 6 & 2 & 2 & 1 \\ \cline{2-5}
           & 7 & 1 & 1 & 1 \\ \cline{2-5}
           \hline
        \end{tabular}
      }
\end{table}

\newpage

\subsubsection{前後比較に基づくダイジェスト生成方式の閾値}
「前後比較に基づくダイジェスト生成方式」では式\ref{fig:zengo}において閾値は, 増加率($Th_{inc}$), 減少率($Th_{dec}$), 一定期間($T$)であった. 図\ref{fig:digest2-1}は前後比較に基づくダイジェスト生成方式の抽出例である. 時刻に付いている赤い丸が正解シーンであり, 縦に緑の線がシステムがダイジェストであると判断して選出した箇所である. 図を見ると2箇所正解していることがわかる.

表\ref{tbl:digest2-1}に増加率0.05の時の正解シーン選出の結果を, 表\ref{tbl:digest2-2}に増加率0.10の時の正解シーン選出の結果を示す. 実験の結果, 増加率が0.1, 減少率が0.03, 一定期間が4である時が一番正解数が多く, 適切な値であることがわかった.

\begin{figure}[h]
  \centering
  \includegraphics[width=1\hsize]{fig/digest2-1.eps}
  \caption{前後比較に基づくダイジェスト生成方式の抽出例}
  \label{fig:digest2-1}
\end{figure}

\begin{table}[h]
  \caption{増加率0.05の時の正解シーン選出の結果}
  \label{tbl:digest2-1}
  \centering
      {\small
        \begin{tabular}{|l|l||l|l|l|} \hline
          \multicolumn{5}{|c|}{増加率($Th_{inc}$)0.05} \\ \hline
          & & \multicolumn{3}{c|}{減少率($Th_{dec}$)} \\ \hline
          & & 0.03 & 0.05 & 0.10 \\ \hline \hline \cline{2-5}
          & 1 & 0 & 0 & 0 \\ \cline{2-5}
          & 2 & 1 & 1 & 1 \\ \cline{2-5}
          & 3 & 0 & 1 & 0 \\ \cline{2-5}
          一定期間($T$)& 4 & 0 & 0 & 1 \\ \cline{2-5}
          & 5 & 0 & 0 & 0 \\ \cline{2-5}
          & 6 & 0 & 0 & 0 \\ \cline{2-5}
          & 7 & 0 & 0 & 1 \\ \cline{2-5}
          \hline
        \end{tabular}
      }
\end{table}

\begin{table}[h]
  \caption{増加率0.10の時の正解シーン選出の結果}
  \label{tbl:digest2-2}
  \centering
      {\small
        \begin{tabular}{|l|l||l|l|l|} \hline
          \multicolumn{5}{|c|}{増加率($Th_{inc}$)0.10} \\ \hline
          & & \multicolumn{3}{c|}{減少率($Th_{dec}$)} \\ \hline
          & & 0.03 & 0.05 & 0.10 \\ \hline \hline \cline{2-5}
          & 1 & 0 & 0 & 0 \\ \cline{2-5}
          & 2 & 1 & 1 & 0 \\ \cline{2-5}
          & 3 & 1 & 1 & 0 \\ \cline{2-5}
          一定期間($T$)& 4 & 2 & 1 & 0 \\ \cline{2-5}
          & 5 & 0 & 1 & 1 \\ \cline{2-5}
          & 6 & 0 & 0 & 0 \\ \cline{2-5}
          & 7 & 0 & 0 & 1 \\ \cline{2-5}
          \hline
        \end{tabular}
      }
\end{table}

\newpage

\subsubsection{最小二乗法に基づくダイジェスト生成方式}
「最小二乗法に基づくダイジェスト生成方式」では式\ref{fig:jijo}において閾値は, 角度(度), データ数(個)であった. 図\ref{fig:digest3-1}は最小二乗法に基づくダイジェスト生成方式の抽出例である. 時刻に付いている赤い丸が正解シーンであり, 縦に緑の線がシステムがダイジェストであると判断して選出した箇所である. 図を見ると5箇所正解していることがわかる.

表\ref{tbl:digest3-1}に正解シーン選出の結果を示す. 実験の結果, 角度(度)55, データ数(個)3である時が一番正解数が多く, 適切な値であることがわかった.

\begin{figure}[h]
  \centering
  \includegraphics[width=1\hsize]{fig/digest3-1.eps}
  \caption{最小二乗法に基づくダイジェスト生成方式の抽出例}
  \label{fig:digest3-1}
\end{figure}

\begin{table}[h]
  \caption{正解シーン選出の結果}
  \label{tbl:digest3-1}
  \centering
      {\small
        \begin{tabular}{|l|l||l|l|l|l|l|l|l|l|l|l|l|l|} \hline
          & & \multicolumn{12}{|c|}{角度(度)} \\ \hline
          & & 25 & 30 & 35 & 40 & 45 & 50 & 55 & 60 & 65 & 70 & 75 & 80 \\ \hline \hline
          & 2 & 0 & 0 & 0 & 2 & 3 & 3 & 2 & 1 & 1 & 0 & 0 & 0 \\ \cline{2-14}
          データ数(個) & 1 & 1 & 1 & 1 & 1 & 1 & 3 & 3 & 2 & 2 & 2 & 2 & 2 \\ \cline{2-14}
          & 0 & 0 & 0 & 0 & 0 & 0 & 0 & 2 & 2 & 2 & 2 & 2 & 2 \\ \cline{2-14}
          & 0 & 0 & 0 & 0 & 0 & 0 & 1 & 1 & 1 & 1 & 1 & 1 & 1 \\ \cline{2-14}
          \hline
        \end{tabular}
      }
\end{table}

\subsection{提案方式の評価}
今までの実験により, それぞれの提案方式の適切な閾値がわかった. 次に, その適切な閾値を適応したそれぞれの提案方式を比較し, P2Pライブストリーミングにおいてダイジェストを生成する際に最も適当なダイジェスト生成方式を決定する.

3つの動画に対して, それぞれの提案方式の適合率を計算することによって評価する. 適合率は式\ref{siki:tekigo}に従う. これは, 提案方式がダイジェストとして選出したもののうち, 元々用意しておいた正解シーンが含まれている確立を意味する. 適合率の値が高いほど, より適切なダイジェスト生成方式であると言える. 表\ref{tbl:tekigo}に実験の結果を示す. 実験の結果, 最小二乗法に基づくダイジェスト生成方式の適合率が最も高く, 平均で66\%を超えるという結果になった.

\begin{eqnarray}
  適合率 = \frac{正解シーン数}{方式による抽出シーン数}
  \label{siki:tekigo}
\end{eqnarray}

\begin{table}[h]
  \caption{適合率の結果}
  \label{tbl:tekigo}
  \centering
      {\small
        \begin{tabular}{|l|l|l|l|} \hline
          & \multicolumn{3}{|c|}{適合率} \\ \hline
          生成方式 & 閾値に基づく方式 & 前後比較に基づく方式 & 最小二乗法に基づく方式 \\ \hline
          動画1 & 12.5\% & 50.0\% & 62.5\% \\ \hline
          動画2 & 37.5\% & 50.0\% & 75.0\% \\ \hline
          動画3 & 50.0\% & 50.0\% & 62.5\% \\ \hline
        \end{tabular}
      }
\end{table}

\subsection{ダイジェスト生成方式に対する評価の考察}
ダイジェスト生成方式の実験では, まずそれぞれの提案方式の適切な閾値を決定した. 次にその閾値を適応した提案方式の適合率を計算することによって最も適切なダイジェスト生成方式を決定した.

適切な閾値の決定では,


\section{トポロジ設計に対する評価}
シミュレーションにより評価

\subsection{実験環境}
実験環境の表、cpu,os

\subsubsection{NS-2}
NS-2の説明、tcl/tkの説明

\subsubsection{goddard}
goddardの説明

\subsection{前提条件}
帯域幅分布、コメント数分布、ノード数とクラスタ

それぞれ表

参加離脱を考慮、複数回参加と離脱を行う、参加と離脱はノードにパケットを送ったかどうかで判断

ノード数が200の時の、役割を与えた場合と与えない場合のトポロジ図

\subsection{適切な役割の割合の決定}
閾値を決定する

\subsubsection{ダイジェストノードの割合}
グラフと結果

\subsubsection{ゲートノードとセミゲートノードの割合}
グラフと結果

\subsection{役割の適切性に対する評価}
役割を与えた場合のスループットの推移のグラフ

役割を与えない場合のスループットの推移のグラフ

役割を与えた場合と与えない場合のスループットの推移の比較のグラフ

\subsection{ダイジェスト保有率に対する評価}
一定時間経つとダイジェスト未取得ノードはダイジェスト取得ノードになる

ダイジェスト保有率の式

ダイジェスト保有率の結果の表

\subsection{トポロジ設計に対する評価の考察}
役割を与えることの有用性確認できた、接続数を分散できたため

ダイジェスト保有率は高かった、コメント数の多いノードにダイジェストをもたせたため


