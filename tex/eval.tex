\section{ダイジェスト生成方式に対する評価}
\subsection{準備}
動画の長さを評価、閾値が存在

\subsection{適切な閾値の決定}
閾値決定方法

\subsubsection{閾値に基づくダイジェスト生成方式の閾値}
閾値

表

\subsubsection{前後比較に基づくダイジェスト生成方式の閾値}
閾値

表

\subsubsection{最小二乗法に基づくダイジェスト生成方式}
閾値

表

\subsection{提案方式の評価}
適合率の式

表

\subsection{ダイジェスト生成方式に対する評価の考察}
案3がよい、誤検知が少ない


\section{トポロジ設計に対する評価}
シミュレーションにより評価

\subsection{実験環境}
実験環境の表、cpu,os

\subsubsection{NS-2}
NS-2の説明、tcl/tkの説明

\subsubsection{goddard}
goddardの説明

\subsection{前提条件}
帯域幅分布、コメント数分布、ノード数とクラスタ

それぞれ表

参加離脱を考慮、複数回参加と離脱を行う、参加と離脱はノードにパケットを送ったかどうかで判断

ノード数が200の時の、役割を与えた場合と与えない場合のトポロジ図

\subsection{適切な役割の割合の決定}
閾値を決定する

\subsubsection{ダイジェストノードの割合}
グラフと結果

\subsubsection{ゲートノードとセミゲートノードの割合}
グラフと結果

\subsection{役割の適切性に対する評価}
役割を与えた場合のスループットの推移のグラフ

役割を与えない場合のスループットの推移のグラフ

役割を与えた場合と与えない場合のスループットの推移の比較のグラフ

\subsection{ダイジェスト保有率に対する評価}
一定時間経つとダイジェスト未取得ノードはダイジェスト取得ノードになる

ダイジェスト保有率の式

ダイジェスト保有率の結果の表

\subsection{トポロジ設計に対する評価の考察}
役割を与えることの有用性確認できた、接続数を分散できたため

ダイジェスト保有率は高かった、コメント数の多いノードにダイジェストをもたせたため


