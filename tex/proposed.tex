ここに提案手法を書く、最初に研究の目的を示す

まず最初にダイジェスト生成方式、次にトポロジ設計
\section{ダイジェスト生成方式}
ダイジェスト生成方式について述べる

3つ提案した

いづれも突出点を発見する手法

\subsection{ダイジェスト生成方式に対する要求条件}
要求条件を提示する

\subsection{閾値に基づくダイジェスト生成方式}
具体的な方法

式

グラフ

\subsection{前後比較に基づくダイジェスト生成方式}
具体的な方法

式

グラフ


\subsection{最小二乗法に基づくダイジェスト生成方式}
具体的な方法

式

グラフ

\subsection{既存のダイジェスト生成方式との比較}

既存手法との定性評価

\section{トポロジ設計}
ダイジェストを生成し、保持し、広めるためのトポロジ設計

\subsection{トポロジ設計に対する要求条件}
要求条件を提示する

\subsection{配信内容に対するコメント}
コメントの役割を書く

\subsection{ノードの役割}
\subsubsection{ゲートノード}
ゲートノードについて

\subsubsection{セミゲートノード}
セミゲートノードについて

\subsubsection{ダイジェストノード}
ダイジェストノードについて

\subsubsection{その他のノード}
トラッカーサーバの存在とノーマルノードについて

\subsection{ノードの役割決定方法}
各役割についての決定方法を書く

役割決定の図

\subsection{ノード間接続方法}

図を説明しながらノード間の接続方法について書く

クラスタ間の図

\subsection{新規参加ピアについて}
新規参加ピアの行動について書く

新規参加ピアの行動のシーケンス図

\subsection{再構築のタイミング}
再構築のタイミング3パータンを書く

場合によっては図を用いる

\subsection{既存のトポロジ設計との比較}

既存手法との定性評価


